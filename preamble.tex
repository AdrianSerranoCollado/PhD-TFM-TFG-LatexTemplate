%\usepackage[a4,cam,center]{crop}
%\crop[font=\upshape\mdseries\small\textsf]

\newcommand{\clearemptydoublepage}{\newpage{\pagestyle{empty}\cleardoublepage}}


\newcommand{\circulo}{\large $\circ$}
\newcommand{\asterisco}{$\ast$}
\newcommand{\cuadrado}{\tiny $\square$}
\newcommand{\triangulo}{\scriptsize $\vartriangle$}
\newcommand{\triangv}{\scriptsize $\triangledown$}
\newcommand{\diamante}{\large $\diamond$}

\usepackage[latin1]{inputenc} % Para poder escribir con acentos y �.
\usepackage[T1]{fontenc}      % Para que haga bien la ``hyphenation''. No
                              % usar si no es necesario, porque ralentiza muchisimo la compilaci�n.
\usepackage{ae}               % Para que todas las fuentes sean Type1, y ninguna Type3.
\usepackage[spanish, english]{babel}

\usepackage{listings}
\usepackage{longtable}
\usepackage{afterpage}

\usepackage{xspace}
\usepackage{verbatim}
\usepackage{moreverb}
\usepackage{multicol}
\usepackage{amsmath}
\usepackage{eurosym}
\usepackage{subfigure}
\usepackage{multirow}
\usepackage{fancyhdr}
\usepackage{makeidx}
\usepackage{rotating}
\usepackage{supertabular}
\usepackage{hhline}
\usepackage{array}
\usepackage[noadjust]{cite}      % Written by Donald Arseneau
                        % V1.6 and later of IEEEtran pre-defines the format
                        % of the cite.sty package \cite{} output to follow
                        % that of IEEE. Loading the cite package will
                        % result in citation numbers being automatically
                        % sorted and properly "ranged". i.e.,
                        % [1], [9], [2], [7], [5], [6]
                        % (without using cite.sty)
                        % will become:
                        % [1], [2], [5]--[7], [9] (using cite.sty)
                        % cite.sty's \cite will automatically add leading
                        % space, if needed. Use cite.sty's noadjust option
                        % (cite.sty V3.8 and later) if you want to turn this
                        % off. cite.sty is already installed on most LaTeX
                        % systems. The latest version can be obtained at:
                        % http://www.ctan.org/tex-archive/macros/latex/contrib/supported/cite/


\usepackage{eso-pic}
\usepackage{graphicx,amssymb,array,psfrag}
\usepackage[pdftex]{epsfig}
\usepackage{amsmath}

% \usepackage{psfrag}
% \psfrag{30000}[][]{\footnotesize $30000$}
% \psfrag{20000}[][]{\footnotesize $20000$}
% \psfrag{10000}[][]{\footnotesize $10000$}
% \psfrag{8000}[][]{\footnotesize $8000$}
% \psfrag{6000}[][]{\footnotesize $6000$}
% \psfrag{5000}[][]{\footnotesize $5000$}
% \psfrag{4500}[][]{\footnotesize $4500$}
% \psfrag{4000}[][]{\footnotesize $4000$}
% \psfrag{3500}[][]{\footnotesize $3500$}
% \psfrag{3000}[][]{\footnotesize $3000$}
% \psfrag{2500}[][]{\footnotesize $2500$}
% \psfrag{2000}[][]{\footnotesize $2000$}
% \psfrag{1800}[][]{\footnotesize $1800$}
% \psfrag{1600}[][]{\footnotesize $1600$}
% \psfrag{1500}[][]{\footnotesize $1500$}
% \psfrag{1400}[][]{\footnotesize $1400$}
% \psfrag{1200}[][]{\footnotesize $1200$}
% \psfrag{1000}[][]{\footnotesize $1000$}
% \psfrag{800}[][]{\footnotesize $800$}
% \psfrag{700}[][]{\footnotesize $700$}
% \psfrag{600}[][]{\footnotesize $600$}
% \psfrag{500}[][]{\footnotesize $500$}
% \psfrag{400}[][]{\footnotesize $400$}
% \psfrag{300}[][]{\footnotesize $300$}
% \psfrag{200}[][]{\footnotesize $200$}
% \psfrag{150}[][]{\footnotesize $150$}
% \psfrag{120}[][]{\footnotesize $120$}
% \psfrag{100}[][]{\footnotesize $100$}
% \psfrag{80}[][]{\footnotesize $80$}
% \psfrag{60}[][]{\footnotesize $60$}
% \psfrag{50}[][]{\footnotesize $50$}
% \psfrag{40}[][]{\footnotesize $40$}
% \psfrag{35}[][]{\footnotesize $35$}
% \psfrag{30}[][]{\footnotesize $30$}
% \psfrag{25}[][]{\footnotesize $25$}
% \psfrag{20}[][]{\footnotesize $20$}
% \psfrag{18}[][]{\footnotesize $18$}
% \psfrag{16}[][]{\footnotesize $16$}
% \psfrag{15}[][]{\footnotesize $15$}
% \psfrag{14}[][]{\footnotesize $14$}
% \psfrag{12}[][]{\footnotesize $12$}
% \psfrag{10}[][]{\footnotesize $10$}
% \psfrag{9}[][]{\footnotesize $9$}
% \psfrag{8}[][]{\footnotesize $8$}
% \psfrag{7}[][]{\footnotesize $7$}
% \psfrag{6}[][]{\footnotesize $6$}
% \psfrag{5}[][]{\footnotesize $5$}
% \psfrag{4}[][]{\footnotesize $4$}
% \psfrag{3}[][]{\footnotesize $3$}
% \psfrag{2}[][]{\footnotesize $2$}
% \psfrag{1.8}[][]{\footnotesize $1.8$}
% \psfrag{1.6}[][]{\footnotesize $1.6$}
% \psfrag{1.5}[][]{\footnotesize $1.5$}
% \psfrag{1.4}[][]{\footnotesize $1.4$}
% \psfrag{1.2}[][]{\footnotesize $1.2$}
% \psfrag{1}[][]{\footnotesize $1$}
% \psfrag{0.9}[][]{\footnotesize $0.9$}
% \psfrag{0.8}[][]{\footnotesize $0.8$}
% \psfrag{0.7}[][]{\footnotesize $0.7$}
% \psfrag{0.6}[][]{\footnotesize $0.6$}
% \psfrag{0.55}[][]{\footnotesize $0.55$}
% \psfrag{0.5}[][]{\footnotesize $0.5$}
% \psfrag{0.45}[][]{\footnotesize $0.45$}
% \psfrag{0.4}[][]{\footnotesize $0.4$}
% \psfrag{0.35}[][]{\footnotesize $0.35$}
% \psfrag{0.3}[][]{\footnotesize $0.3$}
% \psfrag{0.25}[][]{\footnotesize $0.25$}
% \psfrag{0.2}[][]{\footnotesize $0.2$}
% \psfrag{0.15}[][]{\footnotesize $0.15$}
% \psfrag{0.12}[][]{\footnotesize $0.12$}
% \psfrag{0.1}[][]{\footnotesize $0.1$}
% \psfrag{0.08}[][]{\footnotesize $0.08$}
% \psfrag{0.06}[][]{\footnotesize $0.06$}
% \psfrag{0.05}[][]{\footnotesize $0.05$}
% \psfrag{0.04}[][]{\footnotesize $0.04$}
% \psfrag{0.03}[][]{\footnotesize $0.03$}
% \psfrag{0.02}[][]{\footnotesize $0.02$}
% \psfrag{0}[][]{\footnotesize $0$}
% \psfrag{-0.2}[][]{\footnotesize $-0.2$}
% \psfrag{-0.4}[][]{\footnotesize $-0.4$}
% \psfrag{-0.5}[][]{\footnotesize $-0.5$}
% \psfrag{-0.6}[][]{\footnotesize $-0.6$}
% \psfrag{-0.8}[][]{\footnotesize $-0.8$}
% \psfrag{-1}[][]{\footnotesize $-1$}
% \psfrag{-1.5}[][]{\footnotesize $-1.5$}
% \psfrag{-2}[][]{\footnotesize $-2$}
% \psfrag{-3}[][]{\footnotesize $-3$}
% \psfrag{-4}[][]{\footnotesize $-4$}
% \psfrag{-5}[][]{\footnotesize $-5$}
% \psfrag{-6}[][]{\footnotesize $-6$}
% \psfrag{-7}[][]{\footnotesize $-7$}
% \psfrag{-8}[][]{\footnotesize $-8$}
% \psfrag{-9}[][]{\footnotesize $-9$}
% \psfrag{-10}[][]{\footnotesize $-10$}
% \psfrag{-12}[][]{\footnotesize $-12$}
% \psfrag{-14}[][]{\footnotesize $-14$}
% \psfrag{-15}[][]{\footnotesize $-15$}
% \psfrag{-16}[][]{\footnotesize $-16$}
% \psfrag{-18}[][]{\footnotesize $-18$}
% \psfrag{-20}[][]{\footnotesize $-20$}
% \psfrag{-25}[][]{\footnotesize $-25$}
% \psfrag{-30}[][]{\footnotesize $-30$}
% \psfrag{-35}[][]{\footnotesize $-35$}
% \psfrag{-40}[][]{\footnotesize $-40$}
% \psfrag{-45}[][]{\footnotesize $-45$}
% 
% \psfrag{pi/4}[][]{\footnotesize $\pi/4$}
% \psfrag{pi/2}[][]{\footnotesize $\pi/2$}
% \psfrag{3pi/4}[][]{\footnotesize $3\pi/4$}
% \psfrag{pi}[][]{\footnotesize $\pi$}

%\usepackage[margin=10pt,font=small,labelfont=bf,format=hang]
% {caption}[2004/05/16]
% \usepackage[captionskip=15pt]{subfig}
%\usepackage{subfigure}
%\usepackage{url}
%\usepackage{stfloats}
% \usepackage{amsmath}
% \usepackage{amssymb}
% \usepackage{mathrsfs}
% %\usepackage{latexsym}
% \usepackage{verbatim}
%\usepackage{bookman | newcent | palatino | times}  %%% Elegir fuente
%distinta de ``computer roman''.

\usepackage{color}
\definecolor{azul}{rgb}{0,0,1}
\definecolor{verde}{rgb}{0,0.5,0}
\definecolor{rojo}{rgb}{1,0,0}
\definecolor{cyan}{rgb}{0,0.75,0.75}
\definecolor{magenta}{rgb}{0.75,0,0.75}
\definecolor{amarillo}{rgb}{0.75,0.75,0}
\definecolor{gris}{rgb}{0.25,0.25,0.25}
\definecolor{r}{rgb}{0,0,1}
\definecolor{g}{rgb}{0,1,0}
\definecolor{b}{rgb}{1,0,0}
\definecolor{c}{rgb}{0,1,1}
\definecolor{m}{rgb}{1,0,1}
\definecolor{y}{rgb}{1,1,0}
\definecolor{w}{rgb}{1,1,1}
\definecolor{k}{rgb}{0,0,0}
\definecolor{azulE}{rgb}{0,0.3984,0.5977}
\definecolor{amarilloE}{rgb}{0.9961,0.7969,0}
\definecolor{blanco}{rgb}{1,1,1}
\definecolor{burdeos}{rgb}{1,0,0.95}
%%%\usepackage{setspace}
%%%\onehalfspacing

%\usepackage[authoryear]{natbib}
\makeatletter
\let\NAT@parse\undefined
\makeatother
\usepackage{natbib}

\usepackage{geometry}
 \geometry{verbose,a4paper,tmargin=2.5cm,bmargin=2.5cm,lmargin=2.5cm,rmargin=2.5cm}
%\usepackage{indentfirst}
%\geometry{bindingoffset=1cm}

%\usepackage[ps2pdf]{thumbpdf}      %%% thumbnails for ps2pdf (automaticos
%desde Acrobat 5.0)
%\usepackage{geometry}
%\geometry{paperheight=100mm,paperwidth=100mm}
\geometry{paperwidth=210mm,paperheight=297mm}
%\geometry{bindingoffset=1cm}
%\geometry{paperheight=210mm,paperwidth=297mm}
%\usepackage[a3,axes,cam,center]{crop}
%\newcommand*\cropfont[1]{\small\textup{\textmd{\textsf{#1}}}}
%\crop[font=cropfont]
%%\crop[font=\upshape\mdseries\small\textsf] % after GEOMETRY package!!

\usepackage[ps2pdf,                %%% hyper-references for ps2pdf
bookmarks=true,%                   %%% generate bookmarks ...
bookmarksnumbered=true,%           %%% ... with numbers
hypertexnames=false,%              %%% needed for correct links to figures!!!
breaklinks=true,%                  %%% breaks lines, but links are very small
linktocpage=true,                  %%% enlace en el numero de p�gina.
colorlinks=true,linkcolor=azul,    %%%
citecolor=verde,urlcolor=k,        %%% texto  con color
linkbordercolor={0 0 1},%          %%% blue frames around links
pdfborder={0 0 112.0}              %%% border-width of frames 
]{hyperref}%                       %%% will be multiplied with 0.009 by ps2pdf


% Para numerar las \subsubsection
\setcounter{secnumdepth}{5}
%para hacer que las \subsubsection aparezcan en el indice
\setcounter{tocdepth}{5}
%\setcounter{lofdepth}{2}
\setcounter{table}{1}
\setcounter{figure}{1}
\setcounter{secnumdepth}{4}


\setlength{\parskip}{1ex plus 0.5ex minus 0.2ex}


\usepackage{multirow}

\usepackage{setspace}
% \renewcommand{\baselinestretch}{10}
\newcommand{\mycaptiontable}[1]{
\begin{spacing}{0.6}
  %\vspace{0.5cm}
  \begin{quote}
    %\begin{center}
            {{Table} \thechapter.\arabic{table}: #1}
    %\end{center}
  \end{quote}
  %\vspace{1cm}
  \end{spacing}
  \stepcounter{table}
}

\newcommand{\mycaptionfigure}[1]{
  %\vspace{0.5cm}
  \begin{spacing}{0.6}
  \begin{quote}
    %\begin{center}
            {{Figure} \thechapter.\arabic{figure}: #1}
    %\end{center}
  \end{quote}
  %\vspace{1cm}
  \end{spacing}
  \stepcounter{figure}
}

% \newlength{\myVSpace}% the height of the box
% \setlength{\myVSpace}{0ex}% the default, 
% \newcommand\xstrut{\raisebox{-.5\myVSpace}% symmetric behaviour, 
%   {\rule{0pt}{\myVSpace}}%
% }


\sloppy  %better line breaks

\newcommand{\new}[1]{\textcolor{magenta}{#1 }}

\usepackage{amsmath}
%\newcommand{\argmax}{\operatornamewithlimits{argmax}}
%\DeclareMathOperator*{\argmax}{arg\,max}
\newcommand{\argmax}[1]{\underset{#1}{\operatorname{argmax}}}
\newcommand{\argmaxmax}[1]{\underset{#1}{\operatorname{argmaxmax}}}

\usepackage{fancyhdr}

\pagestyle{fancy}

\renewcommand{\chaptermark}[1]{\markboth{\chaptername\ \thechapter.\ #1}{}}
\renewcommand{\sectionmark}[1]{\markright{\thesection\ #1}{}}
\fancyhf{}

\fancyhead[LE,RO]{\bfseries\thepage}
\fancyhead[LO]{\bfseries\rightmark}
\fancyhead[RE]{\bfseries\leftmark}
\renewcommand{\headrulewidth}{0.5pt}
\renewcommand{\footrulewidth}{0pt}
\addtolength{\headheight}{3.5pt}
\fancypagestyle{plain}{\fancyhead{}  \renewcommand{\headrulewidth}{0pt}}

\usepackage{courier}
%***************************************************************************
%***************************************************************************
%***************************************************************************
\usepackage{multirow}
\usepackage{rotating}
\usepackage{setspace, amssymb, amsmath, epsfig, multirow, colortbl, tabularx, subfigure}
\usepackage{acronym}

\onehalfspacing
%\doublespacing
%\DeclareGraphicsExtensions{.eps}


\ifthenelse{\boolean{english}}
{
}
{
\renewcommand{\tablename}{Tabla}
\renewcommand{\listtablename}{�ndice de tablas}
}



%%% Local Variables: 
%%% mode: latex
%%% TeX-master: "tesis"
%%% End: 

% LocalWords:  master
