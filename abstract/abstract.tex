\chapter*{Abstract}
\label{cha:abstract}

\addcontentsline{toc}{chapter}{Abstract}

The work carried out in this Thesis have been focused on the
improvement of the response generation module by the incorporation of
emotional speech synthesis in Spanish. This Thesis is divided in three
stages, each one related with one of the defined scientific
contributions.

Initially, in order to convey emotions through the speech signal, the
relevance of each speech component has been studied. The complementary
behaviour of segmental and supra-segmental rubrics has been
demonstrated, by analysing its relevance for each of the studied
emotions. The nature of the emotions, using an existing corpus, has
been studied using automatic identification strategies and a
perceptual evaluation of emotional stimuli synthesised by
copy-synthesis. In addition to this, a speaker-independent modelling
of emotional acoustic patterns has been studied by means of the
implementation and evaluation of a multi-speaker and multi-language
automatic emotion identification system. Additionally, the performance
of a system for the automatic identification of real emotions (based
on dynamic Bayesian networks) has been evaluated on the first
international emotion recognition challenge.

Secondly, the conclusions obtained from the previous analysis have
been the base for the acquisition of a novel emotional corpus in
Spanish, due to its multimedia and multi-speaker content. This corpus
has been essential for the adaptation and the exhaustive evaluation of
two of the state-of-the-art high quality speech synthesis techniques
to the synthesis of emotional speech: unit selection synthesis, the
dominant technique during last decade; and HMM-based synthesis, an
emerging technique and base of the future research in this area for
the next decade. After, an exhaustive and novel analysis of the
obtained results from a perceptual evaluation, it has been shown that
both techniques synthesise emotional speech with the same
quality. Although the emotions are best identified when they are
synthesised using the unit selection technique and the resulting
emotional strength with this technique is the highest , the HMM-based
synthesis is the technique that best models the prosodic information,
extremely important in expressive speech. The HMM-based system adapted
to Spanish has been awarded as the best system in the text-to-speech
challenge at the Jornadas de Tecnolog�a del Habla in 2008.

Finally, a new strategy for the emotional speaker-independent
transformation of synthetic speech has been designed, implemented and
evaluated using the emotional voices generated with one of the
previous techniques (specifically, the voices successfully generated
using the HMM-based techniques, due to the flexibility and the
controllability of the speech model parameters and the excellent
results obtained in the challenge). This new strategy consists on the
extrapolation of the emotions through the relevant speech components
found in the initial analysis. From the results of the perceptual
evaluation, it has been confirmed that the emotional acoustic patters
have been partially extrapolated to the neutral voice of a target
speaker, without extrapolating the identity of the source
speaker. Additionally, the strength of the extrapolation can be
successfully modified by using an extrapolation factor. However, the
strength of the extrapolation has a negative impact in the quality of
the synthesised speech, especially when the emotion extrapolation is
focused on the transformation of the spectral parameters. Finally, a
new metric for the evaluation of the goodness of the proposed new
strategy has been defined, based on the speech quality, emotion
identification and speaker identification results.

