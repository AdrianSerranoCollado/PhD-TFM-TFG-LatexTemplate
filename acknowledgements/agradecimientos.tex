%%%%%%%%%%%%%%%%%%%%%%%%%%%%%%%%%%%%%%%%%%%%%%%%%%%%%%%%%%%%%%%%%%%%%%%%%%%
%
% Generic template for TFC/TFM/TFG/Tesis
%
% $Id: agradecimientos.tex,v 1.4 2014/01/08 22:56:03 macias Exp $
%
% By:
%  + Javier Mac�as-Guarasa. 
%    Departamento de Electr�nica
%    Universidad de Alcal�
%  + Roberto Barra-Chicote. 
%    Departamento de Ingenier�a Electr�nica
%    Universidad Polit�cnica de Madrid   
% 
% Based on original sources by Roberto Barra, Manuel Oca�a, Jes�s Nuevo,
% Pedro Revenga, Fernando Herr�nz and Noelia Hern�ndez. Thanks a lot to
% all of them, and to the many anonymous contributors found (thanks to
% google) that provided help in setting all this up.
%
% See also the additionalContributors.txt file to check the name of
% additional contributors to this work.
%
% If you think you can add pieces of relevant/useful examples,
% improvements, please contact us at (macias@depeca.uah.es)
%
% Copyleft 2013
%
%%%%%%%%%%%%%%%%%%%%%%%%%%%%%%%%%%%%%%%%%%%%%%%%%%%%%%%%%%%%%%%%%%%%%%%%%%%

\ifthenelse{\equal{\mybooklanguage}{english}}
{
  \chapter*{Acknowledgements}
  \label{cha:acknowledgements}
  \markboth{Acknowledgements}{Acknowledgements}
}
{
  \chapter*{Agradecimientos}
  \label{cha:agradecimientos}
  \markboth{Agradecimientos}{Agradecimientos}
}

% Use this if you don't like the fancy style
\thispagestyle{myplain}



\begin{FraseCelebre}
  \begin{Frase}
    A todos los que la presente vieren y entendieren.
  \end{Frase}
  \begin{Fuente}
    Inicio de las Leyes Org�nicas. Juan Carlos I
  \end{Fuente}
\end{FraseCelebre}

% ``M�s vale un minuto de ilusi�n que mil horas de
% razonamiento''... (cortes�a de Roberto Barra)


Este trabajo es el fruto de muchas horas de trabajo, tanto de los
autores �ltimos de los ficheros de la distribuci�n como de todos los que
en mayor o menor medida han participado en �l a lo largo de su proceso
de gestaci�n.

Menci�n especial merece Manuel Oca�a, el autor de la primera versi�n de
las plantillas de proyectos fin de carrera y tesis doctorales usadas en
el Departamento de Electr�nica de la Universidad de Alcal�, con
contribuciones de Jes�s Nuevo, Pedro Revenga, Fernando Herr�nz y Noelia
Hern�ndez.

En la versi�n actual, la mayor parte de las definiciones de estilos de
partida proceden de la tesis doctoral de Roberto Barra-Chicote, con lo
que gracias muy especiales para �l.

Tambi�n damos las gracias a \input{additionalContributors.txt} que nos
han proporcionado secciones completas y ejemplos puntuales de sus
proyectos fin de carrera.

Finalmente, hay incontables contribuyentes a esta plantilla, la mayor�a
encontrados gracias a la magia del buscador de Google. Hemos intentado
referenciar los m�s importantes en los fuentes de la plantilla, aunque
seguro que hemos omitido alguno. Desde aqu� les damos las gracias a
todos ellos por compartir su saber con el mundo.


% Back to normal JIC. Use it if you set \pagestyle{myplain} above
%\pagestyle{fancy}

%%% Local Variables:
%%% TeX-master: "../book"
%%% End:


