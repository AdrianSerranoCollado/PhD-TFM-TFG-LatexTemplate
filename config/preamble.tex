%\usepackage[a4,cam,center]{crop}
%\crop[font=\upshape\mdseries\small\textsf]

\usepackage[latin1]{inputenc} % Para poder escribir con acentos y �.
\usepackage[T1]{fontenc}      % Para que haga bien la ``hyphenation''. No
                              % usar si no es necesario, porque ralentiza muchisimo la compilaci�n.
\usepackage{ae}               % Para que todas las fuentes sean Type1, y ninguna Type3.

\usepackage[spanish, english]{babel}

\usepackage{listings}
\usepackage{longtable}
\usepackage{afterpage}

\usepackage{xspace}
\usepackage{verbatim}
\usepackage{moreverb}
\usepackage{multicol}
\usepackage{amsmath}
\usepackage{eurosym}
\usepackage{subfigure}
\usepackage{multirow}
\usepackage{fancyhdr}
\usepackage{makeidx}
\usepackage{rotating}
\usepackage{supertabular}
\usepackage{hhline}
\usepackage{array}
\usepackage[noadjust]{cite}      % Written by Donald Arseneau
                        % V1.6 and later of IEEEtran pre-defines the format
                        % of the cite.sty package \cite{} output to follow
                        % that of IEEE. Loading the cite package will
                        % result in citation numbers being automatically
                        % sorted and properly "ranged". i.e.,
                        % [1], [9], [2], [7], [5], [6]
                        % (without using cite.sty)
                        % will become:
                        % [1], [2], [5]--[7], [9] (using cite.sty)
                        % cite.sty's \cite will automatically add leading
                        % space, if needed. Use cite.sty's noadjust option
                        % (cite.sty V3.8 and later) if you want to turn this
                        % off. cite.sty is already installed on most LaTeX
                        % systems. The latest version can be obtained at:
                        % http://www.ctan.org/tex-archive/macros/latex/contrib/supported/cite/


\usepackage{eso-pic}
\usepackage{graphicx,amssymb,array,psfrag}
\usepackage[pdftex]{epsfig}
\usepackage{epstopdf}


\usepackage{color}

%\usepackage[authoryear]{natbib}
% \makeatletter
% \let\NAT@parse\undefined
% \makeatother
% \usepackage{natbib}

\usepackage{geometry}

\usepackage[
ps2pdf,                %%% hyper-references for ps2pdf
bookmarks=true,%                   %%% generate bookmarks ...
bookmarksnumbered=true,            %%% ... with numbers
hypertexnames=false,               %%% needed for correct links to
                                %%% figures!!!
%hypertexnames=true,               %%% needed for correct links on pagebackrefs!!!
breaklinks=true,                   %%% breaks lines, but links are very small
%pagebackref=true,
%linktocpage=true,                 %%% enlace en el numero de p�gina.
linktoc=all,
colorlinks=true,
linkcolor=azul,    
citecolor=verde,
urlcolor=azul,                     %%% texto  con color
linkbordercolor={0 0 1},           %%% blue frames around links
pdfborder={0 0 112.0}              %%% border-width of frames 
]{hyperref}                        %%% will be multiplied with 0.009 by ps2pdf


% Para numerar las \subsubsection
\setcounter{secnumdepth}{5}
%para hacer que las \subsubsection aparezcan en el indice
\setcounter{tocdepth}{5}
%\setcounter{lofdepth}{2}
\setcounter{table}{1}
\setcounter{figure}{1}
\setcounter{secnumdepth}{4}


\setlength{\parskip}{1ex plus 0.5ex minus 0.2ex}


\usepackage{multirow}

\usepackage{setspace}
% \renewcommand{\baselinestretch}{10}
\newcommand{\mycaptiontable}[1]{
\begin{spacing}{0.6}
  %\vspace{0.5cm}
  \begin{quote}
    %\begin{center}
            {{Table} \thechapter.\arabic{table}: #1}
    %\end{center}
  \end{quote}
  %\vspace{1cm}
  \end{spacing}
  \stepcounter{table}
}

\newcommand{\mycaptionfigure}[1]{
  %\vspace{0.5cm}
  \begin{spacing}{0.6}
  \begin{quote}
    %\begin{center}
            {{Figure} \thechapter.\arabic{figure}: #1}
    %\end{center}
  \end{quote}
  %\vspace{1cm}
  \end{spacing}
  \stepcounter{figure}
}

% \newlength{\myVSpace}% the height of the box
% \setlength{\myVSpace}{0ex}% the default, 
% \newcommand\xstrut{\raisebox{-.5\myVSpace}% symmetric behaviour, 
%   {\rule{0pt}{\myVSpace}}%
% }

\usepackage{amsmath}
\usepackage{fancyhdr}

\usepackage{courier}

%***************************************************************************
%***************************************************************************
%***************************************************************************
\usepackage{multirow}
\usepackage{rotating}
\usepackage{setspace, amssymb, amsmath, epsfig, multirow, colortbl, tabularx, subfigure}
\usepackage{acronym}

% ifthen to allow using language dependent settings
\usepackage{ifthen}

\newcommand{\clearemptydoublepage}{\newpage{\pagestyle{empty}\cleardoublepage}}

\newcommand{\circulo}{\large $\circ$}
\newcommand{\asterisco}{$\ast$}
\newcommand{\cuadrado}{\tiny $\square$}
\newcommand{\triangulo}{\scriptsize $\vartriangle$}
\newcommand{\triangv}{\scriptsize $\triangledown$}
\newcommand{\diamante}{\large $\diamond$}


\newcommand{\new}[1]{\textcolor{magenta}{#1 }}
\newcommand{\argmax}[1]{\underset{#1}{\operatorname{argmax}}}
\newcommand{\argmaxmax}[1]{\underset{#1}{\operatorname{argmaxmax}}}


\pagestyle{fancy}

\definecolor{azul}{rgb}{0,0,1}
\definecolor{verde}{rgb}{0,0.5,0}
\definecolor{rojo}{rgb}{1,0,0}
\definecolor{cyan}{rgb}{0,0.75,0.75}
\definecolor{magenta}{rgb}{0.75,0,0.75}
\definecolor{amarillo}{rgb}{0.75,0.75,0}
\definecolor{gris}{rgb}{0.25,0.25,0.25}
\definecolor{r}{rgb}{0,0,1}
\definecolor{g}{rgb}{0,1,0}
\definecolor{b}{rgb}{1,0,0}
\definecolor{c}{rgb}{0,1,1}
\definecolor{m}{rgb}{1,0,1}
\definecolor{y}{rgb}{1,1,0}
\definecolor{w}{rgb}{1,1,1}
\definecolor{k}{rgb}{0,0,0}
\definecolor{azulE}{rgb}{0,0.3984,0.5977}
\definecolor{amarilloE}{rgb}{0.9961,0.7969,0}
\definecolor{blanco}{rgb}{1,1,1}
\definecolor{burdeos}{rgb}{1,0,0.95}

\geometry{verbose,a4paper,tmargin=2.5cm,bmargin=2.5cm,lmargin=2.5cm,rmargin=2.5cm}
\geometry{paperwidth=210mm,paperheight=297mm}

\providecommand\phantomsection{}
\onehalfspacing
\sloppy  %better line breaks

\renewcommand{\chaptermark}[1]{\markboth{\chaptername\ \thechapter.\ #1}{}}
\renewcommand{\sectionmark}[1]{\markright{\thesection\ #1}{}}

\fancyhf{}

\fancyhead[LE,RO]{\bfseries\thepage}
\fancyhead[LO]{\bfseries\rightmark}
\fancyhead[RE]{\bfseries\leftmark}

\renewcommand{\headrulewidth}{0.5pt}
\renewcommand{\footrulewidth}{0pt}
\addtolength{\headheight}{3.5pt}
\fancypagestyle{plain}{\fancyhead{}  \renewcommand{\headrulewidth}{0pt}}

