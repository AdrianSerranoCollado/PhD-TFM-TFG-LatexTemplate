\title{beamer examples}
\subtitle{created with beamer 3.x}
\author{Matthias Pospiech}
\institute{University of Hannover}
\titlegraphic{}
\date{\today}
% --------------------------------------------------- Slide --
\begin{frame}[plain]
  \titlepage
\end{frame}
% --------------------------------------------------- Slide --
% \section[Contents]{}
% ------------------------------------------------------------
% \begin{frame}
% 	\frametitle{Contents}
% 	\tableofcontents[%
% 		currentsection, % causes all sections but the current to be shown in a semi-transparent way.
% % 		currentsubsection, % causes all subsections but the current subsection in the current section to ...
% % 		hideallsubsections, % causes all subsections to be hidden.
% 		hideothersubsections, % causes the subsections of sections other than the current one to be hidden.
% % 		part=, % part number causes the table of contents of part part number to be shown
% 		pausesections, % causes a \pause command to be issued before each section. This is useful if you
% % 		pausesubsections, %  causes a \pause command to be issued before each subsection.
% % 		sections={ overlay specification },
% 	]
% \end{frame}
% --------------------------------------------------- PART ---
\part{Tutorial}
\frame{\partpage}
% ------------------------------------------------------------
\begin{frame}
	\frametitle{Contents}
	\tableofcontents[%
% 		currentsection, % causes all sections but the current to be shown in a semi-transparent way.
% 		currentsubsection, % causes all subsections but the current subsection in the current section to ...
% 		hideallsubsections, % causes all subsections to be hidden.
% 		hideothersubsections, % causes the subsections of sections other than the current one to be hidden.
% 		part=, % part number causes the table of contents of part part number to be shown
		pausesections, % causes a \pause command to be issued before each section. This is useful if you
% 		pausesubsections, %  causes a \pause command to be issued before each subsection.
% 		sections={ overlay specification },
	]
\end{frame}
% ------------------------------------------------------------
\section{Tutorial: Euclid's Presentation}
% ------------------------------------------------------------
% --------------------------------------------------- Slide --
\subsection{Creating a Simple Frame}
% ------------------------------------------------------------
\begin{frame}
  \frametitle{What Are Prime Numbers?}
  A prime number is a number that has exactly two divisors.
\end{frame}
% --------------------------------------------------- Slide --
\begin{frame}
  \frametitle{What Are Prime Numbers?}
  \begin{definition}
    A \alert{prime number} is a number that has exactly two divisors
  \end{definition}
  \begin{example}
    \begin{itemize}
    \item 2 is prime (two divisors: 1 and 2).
    \item 3 is prime (two divisors: 1 and 3).
    \item 4 is not prime (\alert{three} divisors: 1, 2, and 4).
    \end{itemize}
  \end{example}
\end{frame}
% --------------------------------------------------- Slide --
\subsection{Creating Simple Overlays}
% ------------------------------------------------------------
\begin{frame}
  \frametitle{What Are Prime Numbers?}
  \begin{definition}
    A \alert{prime number} is a number that has exactly two divisors
  \end{definition}
  \begin{example}
		\begin{itemize}
			\item 2 is prime (two divisors: 1 and 2).
  			\pause
			\item 3 is prime (two divisors: 1 and 3).
  			\pause
			\item 4 is not prime (\alert{three} divisors: 1, 2, and 4).
		\end{itemize}
  \end{example}
\end{frame}
% --------------------------------------------------- Slide --
\begin{frame}
  \frametitle{There Is No Largest Prime Number}
  \framesubtitle{The proof uses \textit{reductio ad absurdum}.}
  \begin{theorem}
    There is no largest prime number.
  \end{theorem}
  \begin{proof}
    \begin{enumerate}
    \item<1-> Suppose $p$ were the largest prime number.
    \item<2-> Let $q$ be the product of the first $p$ numbers.
    \item<3-> Then $q + 1$ is not divisible by any of them.
    \item<1-> Thus $q + 1$ is also prime and greater than $p$.\qedhere
    \end{enumerate}
  \end{proof}
  \uncover<4->{The proof used \textit{reductio ad absurdum}.}
\end{frame}
% --------------------------------------------------- Slide --
\subsection{Structuring a Frame}
% ------------------------------------------------------------
\newtheorem{answeredquestions}[theorem]{Answered Questions}
\newtheorem{openquestions}[theorem]{Open Questions}
% ------------------------------------------------------------
\begin{frame}
  \frametitle{What's Still To Do?}
  \begin{block}{Answered Questions}
    How many primes are there?
  \end{block}
  \begin{block}{Open Questions}
    Is every even number the sum of two primes?
  \end{block}
\end{frame}
% --------------------------------------------------- Slide --
\begin{frame}
  \frametitle{What's Still To Do?}
  \begin{itemize}
  \item Answered Questions
    \begin{itemize}
    \item How many primes are there?
    \end{itemize}
  \item Open Questions
    \begin{itemize}
    \item Is every even number the sum of two primes?
    \end{itemize}
  \end{itemize}
\end{frame}
% --------------------------------------------------- Slide --
\begin{frame}
  \frametitle{What's Still To Do?}
  \begin{columns}
    \column{.5\textwidth}
      \begin{block}{Answered Questions}
        How many primes are there?
      \end{block}
    \column{.5\textwidth}
      \begin{block}{Open Questions}
        Is every even number the sum of two primes?
        \cite{Goldbach1742}
      \end{block}
  \end{columns}
\end{frame}
% --------------------------------------------------- Slide --
\subsection{Verbatim Text}
% ------------------------------------------------------------
\begin{frame}[fragile]
  \frametitle{An Algorithm For Finding Primes Numbers.}
	\begin{verbatim}
int main (void)
{
  std::vector<bool> is_prime (100, true);
  for (int i = 2; i < 100; i++)
    if (is_prime[i])
      {
        std::cout << i << " ";
        for (int j = i; j < 100;
            is_prime [j] = false, j+=i);
      }
  return 0;
}
	\end{verbatim}
%  \begin{uncoverenv}<2>
%     Note the use of \verb|std::|.
%  \end{uncoverenv}
\end{frame}
% --------------------------------------------------- Slide --
\begin{frame}[fragile]
  \frametitle{An Algorithm For Finding Primes Numbers.}
\begin{semiverbatim}
\uncover<1->{\alert<0>{int main (void)}}
\uncover<1->{\alert<0>{\{}}
\uncover<1->{\alert<1>{ \alert<4>{std::}vector<bool> is_prime (100, true);}}
\uncover<1->{\alert<1>{ for (int i = 2; i < 100; i++)}}
\uncover<2->{\alert<2>{    if (is_prime[i])}}
\uncover<2->{\alert<0>{      \{}}
\uncover<3->{\alert<3>{        \alert<4>{std::}cout << i << " ";}}
\uncover<3->{\alert<3>{        for (int j = i; j < 100;}}
\uncover<3->{\alert<3>{             is_prime [j] = false, j+=i);}}
\uncover<2->{\alert<0>{      \}}}
\uncover<1->{\alert<0>{ return 0;}}
\uncover<1->{\alert<0>{\}}}
\end{semiverbatim}
  \visible<4->{Note the use of \alert{\texttt{std::}}.}
\end{frame}
% --------------------------------------------------- PART ---
\part{Howtos}
\frame{\partpage}
% ------------------------------------------------------------
\begin{frame}
	\frametitle{Contents}
	\tableofcontents[%
% 		currentsection, % causes all sections but the current to be shown in a semi-transparent way.
% 		currentsubsection, % causes all subsections but the current subsection in the current section to ...
% 		hideallsubsections, % causes all subsections to be hidden.
% 		hideothersubsections, % causes the subsections of sections other than the current one to be hidden.
% 		part=, % part number causes the table of contents of part part number to be shown
		pausesections, % causes a \pause command to be issued before each section. This is useful if you
% 		pausesubsections, %  causes a \pause command to be issued before each subsection.
% 		sections={ overlay specification },
	]
\end{frame}
% ------------------------------------------------------------
\section{How To Uncover Things Piecewise}
% ------------------------------------------------------------
\subsection{Uncovering an Enumeration Piecewise}
% ------------------------------------------------------------
\begin{frame}
\begin{itemize}
\item<1-> First point.
\item<2-> Second point.
\item<3-> Third point.
\end{itemize}

\begin{itemize}[<+->]
\item First point.
\item Second point.
\item Third point.
\end{itemize}

\begin{itemize}[<+->]
\item First point.
\item<.-> Second point.
\item Third point.
\end{itemize}
\end{frame}
% --------------------------------------------------- Slide --
\subsection{Hilighting the Current Item in an Enumeration}
% ------------------------------------------------------------
\begin{frame}
\begin{itemize}
\item<1-| alert@1> First point.
\item<2-| alert@2> Second point.
\item<3-| alert@3> Third point.
\end{itemize}
or
\begin{itemize}[<+-| alert@+>]
\item First point.
\item Second point.
\item Third point.
\end{itemize}
\end{frame}
% --------------------------------------------------- Slide --
\subsection{Changing Symbol Before an Enumeration}
% ------------------------------------------------------------
\newenvironment{ballotenv}
{\only{%
  \setbeamertemplate{itemize item}{code for showing a ballot}%
  \setbeamertemplate{itemize subitem}{code for showing a smaller ballot}%
  \setbeamertemplate{itemize subsubitem}{code for showing a smaller ballot}}}
{}
\begin{frame}
\begin{itemize}
\item<1-| ballot@1> First point.
\item<2-| ballot@2> Second point.
\item<3-| ballot@3> Third point.
\end{itemize}
and
\begin{itemize}[<+-| ballot@+>]
\item First point.
\item Second point.
\item Third point.
\end{itemize}
\end{frame}
% --------------------------------------------------- Slide --
\begin{frame}
In the following example, more and more items become "checked" from slide to slide:
\begin{itemize}[<ballot@+-| visible@1-,+(1)>]
\item First point.
\item Second point.
\item Third point.
\end{itemize}
\end{frame}
% --------------------------------------------------- Slide --
\subsection{Uncovering Piecewise}
% ------------------------------------------------------------
\begin{frame}
Uncovering Tagged Formulas Piecewise
\begin{align}
  A &= B \\
    \uncover<2->{&= C \\}
    \uncover<3->{&= D \\}
    \notag
  \end{align}
\vskip-1.5em
\end{frame}

\begin{frame}
Uncovering a Table Rowwise \newline
\rowcolors[]{1}{blue!20}{blue!10}
\begin{tabular}{l!{\vrule}cccc}
  Class & A & B & C & D \\\hline
  X     & 1 & 2 & 3 & 4 \pause\\
  Y     & 3 & 4 & 5 & 6 \pause\\
  Z     &5&6&7&8
\end{tabular}
\end{frame}

\begin{frame}
Uncovering a Table Columnwise \newline
\rowcolors[]{1}{blue!20}{blue!10}
\begin{tabular}{l!{\vrule}c<{\onslide<2->}c<{\onslide<3->}c<{\onslide<4->}c<{\onslide}c}
  Class & A & B & C & D \\
  X     & 1 & 2 & 3 & 4 \\
  Y     & 3 & 4 & 5 & 6 \\
  Z     &5&6&7&8
\end{tabular}
\end{frame}
% --------------------------------------------------- PART ---
\part{Building a Presentation}
\frame{\partpage}
% ------------------------------------------------------------
\begin{frame}
	\frametitle{Contents}
	\tableofcontents[%
%   	currentsection, % causes all sections but the current to be shown in a semi-transparent way.
% 		currentsubsection, % causes all subsections but the current subsection in the current section to ...
 		hideallsubsections, % causes all subsections to be hidden.
% 		hideothersubsections, % causes the subsections of sections other than the current one to be hidden.
% 		part=, % part number causes the table of contents of part part number to be shown
		pausesections, % causes a \pause command to be issued before each section. This is useful if you
% 		pausesubsections, %  causes a \pause command to be issued before each subsection.
% 		sections={ overlay specification },
	]
\end{frame}
% --------------------------------------------------- Slide --
\section{Creating Overlays}
% ------------------------------------------------------------
\subsection{The Pause Commands}
% ------------------------------------------------------------
\begin{frame}
  \begin{itemize}
  \item
    Shown from first slide on.
  \pause
  \item
    Shown from second slide on.
    \begin{itemize}
    \item
      Shown from second slide on.
    \pause
    \item
      Shown from third slide on.
    \end{itemize}
  \item
    Shown from third slide on.
  \pause
  \item
    Shown from fourth slide on.
  \end{itemize}
  Shown from fourth slide on.
  \begin{itemize}
  \onslide
  \item
    Shown from first slide on.
  \pause
  \item
    Shown from fifth slide on.
  \end{itemize}
\end{frame}
% --------------------------------------------------- Slide --
\subsection{Commands with Overlay Specifications}
% ------------------------------------------------------------
\begin{frame}
  \textbf{This line is bold on all three slides.}
  \textbf<2>{This line is bold only on the second slide.}
  \textbf<3>{This line is bold only on the third slide.}
\end{frame}
% --------------------------------------------------- Slide --
\begin{frame}
  \only<1>{This line is inserted only on slide 1.}
  \only<2>{This line is inserted only on slide 2.}
\end{frame}
% --------------------------------------------------- Slide --
\begin{frame}
  Shown on first slide.
  \onslide<2-3>
  Shown on second and third slide.
  \begin{itemize}
  \item
    Still shown on the second and third slide.
  \onslide+<4->
  \item
    Shown from slide 4 on.
  \end{itemize}
  Shown from slide 4 on.
  \onslide
  Shown on all slides.
\end{frame}
% --------------------------------------------------- Slide --
\begin{frame}
  \onslide<1>{Same effect as the following command.}
  \uncover<1>{Same effect as the previous command.}
  \onslide+<2>{Same effect as the following command.}
  \visible<2>{Same effect as the previous command.}
  \onslide*<3>{Same effect as the following command.}
  \only<3>{Same effect as the previous command.}
\end{frame}
% --------------------------------------------------- Slide --
\begin{frame}
\temporal<3-4>{Shown on 1, 2}{Shown on 3, 4}{Shown 5, 6, 7, ...}
\temporal<3,5>{Shown on 1, 2, 4}{Shown on 3, 5}{Shown 6, 7, 8, ...}
\end{frame}
% --------------------------------------------------- Slide --
\def\colorize<#1>{%
  \temporal<#1>{\color{red!50}}{\color{black}}{\color{black!50}}}
\begin{frame}
  \begin{itemize}
    \colorize<1> \item First item.
    \colorize<2> \item Second item.
    \colorize<3> \item Third item.
    \colorize<4> \item Fourth item.
  \end{itemize}
\end{frame}
% --------------------------------------------------- Slide --
\begin{frame}
  \begin{enumerate}
  \item<3-| alert@3>[0.] A zeroth point, shown at the very end.
  \item<1-| alert@1> The first and main point.
  \item<2-| alert@2> The second point.
  \end{enumerate}
\end{frame}
% --------------------------------------------------- Slide --
\subsection{Environments with Overlay Specifications}
% ------------------------------------------------------------
\begin{frame}
  \frametitle{A Theorem on Infinite Sets}
  \begin{theorem}<1->
    There exists an infinite set.
  \end{theorem}
  \begin{proof}<3->
    This follows from the axiom of infinity.
  \end{proof}
  \begin{example}<2->
    The set of natural numbers is infinite.
  \end{example}
\end{frame}
% --------------------------------------------------- Slide --
\begin{frame}
  This line is always shown.
  \begin{onlyenv}<2>
    This line is inserted on slide 2.
  \end{onlyenv}
\end{frame}
% --------------------------------------------------- Slide --
\begin{frame}
  This
  \begin{altenv}<2>{(}{)}{[}{]}
    word
  \end{altenv}
  is in round brackets on slide 2 and in square brackets on slide 1.
\end{frame}
% --------------------------------------------------- Slide --
\subsection{Dynamically Changing Text or Images}
% ------------------------------------------------------------
\begin{frame}
	\begin{overlayarea}{\textwidth}{3cm}
  		\only<1>{Some text for the first slide.\\ Possibly several lines long.}
  		\only<2>{Replacement on the second slide.}
	\end{overlayarea}
\end{frame}
% --------------------------------------------------- Slide --
\begin{frame}
	\begin{overprint}
  		\onslide<1| handout:1>
    		Some text for the first slide.\\
    		Possibly several lines long.
  		\onslide<2| handout:0>
    		Replacement on the second slide. Supressed for handout.
	\end{overprint}
\end{frame}
% --------------------------------------------------- Slide --
\subsection{Advanced Overlay Specifications}
% ------------------------------------------------------------
\begin{frame}
  \begin{actionenv}<1-| alert@3-4,6>
    This text is shown the same way as the text below.
  \end{actionenv}
  \begin{uncoverenv}<2->
    \begin{alertenv}<3-4,6>
      This text is shown the same way as the text above.
    \end{alertenv}
  \end{uncoverenv}
\end{frame}
% --------------------------------------------------- Slide --
\begin{frame}
	\begin{itemize}
		\item<+-> Apple
		\item<+-> Peach
		\item<+-> Plum
		\item<+-> Orange
	\end{itemize}

	\begin{itemize}[<+-| alert@+>]
		\item Apple
		\item Peach
		\item Plum
		\item Orange
	\end{itemize}

	\begin{itemize}[<+->]
		\item This is \alert<.>{important}.
		\item We want to \alert<.>{highlight} this and \alert<.>{this}.
		\item What is the \alert<.>{matrix}?
	\end{itemize}
\end{frame}
% --------------------------------------------------- Slide --
% \section{Adding Parts}
% % ------------------------------------------------------------
% \frame{\titlepage}
% \section*{Outlines}
% \subsection{Part I: Review of Previous Lecture}
% \frame{
%   \frametitle{Outline of Part I}
%   \tableofcontents[part=1]}
% \subsection{Part II: Today's Lecture}
% \frame{
%   \frametitle{Outline of Part II}
%   \tableofcontents[part=2]}
% \part{Review of Previous Lecture}
% \frame{\partpage}
% \section[Previous Lecture]{Summary of the Previous Lecture}
% \subsection{Topics}
% \frame{...}
% \subsection{Learning Objectives}
% \frame{...}
% \part{Today's Lecture}
% \frame{\partpage}
% \section{Topic A}
% \frame{\tableofcontents[currentsection]}
% \subsection{Foo}
% \frame{...}
% \section{Topic B}
% \frame{\tableofcontents[currentsection]}
% \subsection{bar}
% \frame{...}
% --------------------------------------------------- Slide --
\section{Structuring a Presentation: The Interactive Global Structure}
% ------------------------------------------------------------
\subsection{Adding Hyperlinks and Buttons}
% ------------------------------------------------------------
\begin{frame}
  \begin{itemize}
  \item<1-> First item.
  \item<2-> Second item.
  \item<3-> Third item.
  \end{itemize}
  \hyperlink{jumptosecond}{\beamergotobutton{Jump to second slide}}
  \hypertarget<2>{jumptosecond}{}
\end{frame}
% --------------------------------------------------- Slide --
\begin{frame}[label=threeitems]
  \begin{itemize}
  \item<1-> First item.
  \item<2-> Second item.
  \item<3-> Third item.
  \end{itemize}
  \hyperlink{threeitems<2>}{\beamergotobutton{Jump to second slide}}
\end{frame}
% --------------------------------------------------- Slide --
\frame{
  \begin{theorem}
    ...
  \end{theorem}
  \begin{overprint}
  \onslide<1>
    \hfill\hyperlinkframestartnext{\beamerskipbutton{Skip proof}}
  \onslide<2>
    \begin{proof}
      ...
    \end{proof}
  \end{overprint}
}
% --------------------------------------------------- Slide --
% \frame<1>[label=mytheorem]
% {
%   \begin{theorem}
%     ...
%   \end{theorem}
%   \begin{overprint}
%   \onslide<1>
%     \hfill\hyperlink{mytheorem<2>}{\beamergotobutton{Go to proof details}}
%   \onslide<2>
%     \begin{proof}
%       ...
%     \end{proof}
%     \hfill\hyperlink{mytheorem<1>}{\beamerreturnbutton{Return}}
%   \end{overprint}
% }
% \appendix
% \againframe<2>{mytheorem}
% % --------------------------------------------------- Slide --
% \subsection{Repeating a Frame at a Later Point}
% % ------------------------------------------------------------
% \frame<1-2>[label=myframe]
% {
%   \begin{itemize}
%   \item<alert@1> First subject.
%   \item<alert@2> Second subject.
%   \item<alert@3> Third subject.
%   \end{itemize}
% }
% \frame
% {
%   Some stuff explaining more on the second matter.
% }
% \againframe<3>{myframe}
% --------------------------------------------------- Slide --
% \subsection{Adding Anticipated Zooming}
% % ------------------------------------------------------------
% \begin{frame}<1>[label=zooms]
%   \frametitle<1>{A Complicated Picture}
%   \framezoom<1><2>[border](0cm,0cm)(2cm,1.5cm)
%   \framezoom<1><3>[border](1cm,3cm)(2cm,1.5cm)
%   \framezoom<1><4>[border](3cm,2cm)(3cm,2cm)
%   \pgfimage[height=8cm]{complicatedimagefilename}
% \end{frame}
% \againframe<2->[plain]{zooms}
% --------------------------------------------------- Slide --
\section{Structuring a Presentation: The Local Structure}
% ------------------------------------------------------------
\subsection{Itemizations, Enumerations, and Descriptions}
% ------------------------------------------------------------
\begin{frame}
  There are three important points:
  \begin{enumerate}
  \item<1-> A first one,
  \item<2-> a second one with a bunch of subpoints,
    \begin{itemize}
    \item first subpoint. (Only shown from second slide on!).
    \item<3-> second subpoint added on third slide.
    \item<4-> third subpoint added on fourth slide.
    \end{itemize}
  \item<5-> and a third one.
  \end{enumerate}
\end{frame}
% --------------------------------------------------- Slide --
\begin{frame}
\begin{itemize}[<+->]
\item This is shown from the first slide on.
\item This is shown from the second slide on.
\item This is shown from the third slide on.
\item<1-> This is shown from the first slide on.
\item This is shown from the fourth slide on.
\end{itemize}
\end{frame}
% --------------------------------------------------- Slide --
\begin{frame}
\begin{description}[<+->][longest label]
\item[short] Some text.
\item[longest label] Some text.
\item[long label] Some text.
\end{description}
\end{frame}
% --------------------------------------------------- Slide --
\subsection{Block Environments}
% ------------------------------------------------------------
\begin{frame}
	\begin{block}<1->{Definition}
  		A \alert{set} consists of elements.
	\end{block}
	\begin{alertblock}{Wrong Theorem}
  		$1=2$.
	\end{alertblock}
	\begin{exampleblock}{Example}<only@2->
  		The set $\{1,2,3,5\}$ has four elements.
	\end{exampleblock}
\end{frame}
% --------------------------------------------------- Slide --
\subsection{Theorem Environments}
% ------------------------------------------------------------
\begin{frame}
  \frametitle{A Theorem on Infinite Sets}
  \begin{theorem}<1->
    There exists an infinite set.
  \end{theorem}
  \begin{proof}<2->
    This follows from the axiom of infinity.
  \end{proof}
  \begin{example}<3->[Natural Numbers]
    The set of natural numbers is infinite.
  \end{example}
\end{frame}
% --------------------------------------------------- Slide --
\subsection{Framed and Boxed Text}
% ------------------------------------------------------------
\begin{frame}
	\begin{beamercolorbox}[ht=2.5ex,dp=1ex,center]{title in head/foot}
  		\usebeamerfont{title in head/foot}
  		\insertshorttitle
	\end{beamercolorbox}%
	\begin{beamercolorbox}[ht=2.5ex,dp=1ex,center]{author in head/foot}
  		\usebeamerfont{author in head/foot}
  		\insertshortauthor
	\end{beamercolorbox}
	\mbox{}\medskip\newline
	Typesetting a postit:\newline
	\setbeamercolor{postit}{fg=black,bg=yellow}
	\begin{beamercolorbox}[sep=1em,wd=5cm]{postit}
  		Place me somewhere!
	\end{beamercolorbox}
	\mbox{}\medskip\newline
	\begin{beamerboxesrounded}[upper=block head,lower=block body,shadow=true]{Theorem}
  		$A = B$.
	\end{beamerboxesrounded}
\end{frame}
% --------------------------------------------------- Slide --
\subsection{Splitting a Frame into Multiple Columns}
% ------------------------------------------------------------
\begin{frame}
	\begin{columns}[t]
  		\begin{column}{5cm}
    		Two\\lines.
  		\end{column}
  		\begin{column}{5cm}
    		One line (but aligned).
  		\end{column}
	\end{columns}
\end{frame}
% --------------------------------------------------- Slide --
\section{Animations, Sounds, and Slide Transitions}
% ------------------------------------------------------------
\subsection{Animations Created by Showing Slides in Rapid Succession}
% ------------------------------------------------------------
% \begin{frame}
%   \frametitle{A Five Slide Animation}
%   \animate<2-4>
%   ... code for creating an animation with five slides ...
% \end{frame}
% --------------------------------------------------- Slide --
\newcount\opaqueness
\begin{frame}
  anomations only work in full screen mode in Acrobat Reader !
  \animate<2-10>
  \animatevalue<1-10>{\opaqueness}{100}{0}
  \begin{colormixin}{\the\opaqueness!averagebackgroundcolor}
    \frametitle{Fadeout Frame}
    This text (and all other frame content) will fade out when the
    second slide is shown. This even works with
    {\color{green!90!black}colored} \alert{text}.
  \end{colormixin}
\end{frame}
\newcount\opaqueness
\newdimen\offset
\begin{frame}
  \frametitle{Flying Theorems (You Really Shouldn't!)}
  \animate<2-14>
  \animatevalue<1-15>{\opaqueness}{100}{0}
  \animatevalue<1-15>{\offset}{0cm}{-5cm}
  \begin{colormixin}{\the\opaqueness!averagebackgroundcolor}
  \hskip\offset
    \begin{minipage}{\textwidth}
      \begin{theorem}
        This theorem flies out.
      \end{theorem}
    \end{minipage}
  \end{colormixin}
  \animatevalue<1-15>{\opaqueness}{0}{100}
  \animatevalue<1-15>{\offset}{-5cm}{0cm}
  \begin{colormixin}{\the\opaqueness!averagebackgroundcolor}
  \hskip\offset
    \begin{minipage}{\textwidth}
      \begin{theorem}
        This theorem flies in.
      \end{theorem}
    \end{minipage}
  \end{colormixin}
\end{frame}
% --------------------------------------------------- Slide --
\subsection{Slide Transitions}
% ------------------------------------------------------------
\begin{frame}
	Slide Transitions only work in full screen mode in Acrobat Reader !
  \begin{example}<1,6,11>[examples for Slide Transitions]{This line is shown on each slide of slide transitions}\end{example}
  \begin{example}<2,7,12>[examples for Slide Transitions]{This line is shown on each slide of slide transitions}\end{example}
  \begin{example}<3,8>[examples for Slide Transitions]{This line is shown on each slide of slide transitions}\end{example}
  \begin{example}<4,9>[examples for Slide Transitions]{This line is shown on each slide of slide transitions}\end{example}
  \begin{example}<5,10>[examples for Slide Transitions]{This line is shown on each slide of slide transitions}\end{example}
  \transdissolve<1>[duration=0.2]
  \transblindshorizontal<2>
  \transblindsvertical<3>
  \transboxin<4>
  \transboxout<5>
  \transdissolve<6>[duration=0.2]
  \transglitter<7>[direction=90]
  \transsplitverticalin<8>
  \transsplitverticalout<9>
  \transsplithorizontalin<10>
  \transsplithorizontalout<11>
  \transwipe<12>[direction=90]
\end{frame}

% --------------------------------------------------- Slide --
\section{Adding Notes}
% ------------------------------------------------------------
\begin{frame}
  \begin{itemize}
  \item<1-> Eggs
  \item<2-> Plants
    \note[item]<2>{Tell joke about plants.}
    \note[item]<2>{Make it short.}
  \item<3-> Animals
  \end{itemize}
\end{frame}

% % --------------------------------------------------- Slide --
% \begin{frame}
%
% \end{frame}

% ------------------------------------------------------------
\begin{thebibliography}{10}
\bibitem{Goldbach1742}[Goldbach, 1742]
  Christian Goldbach.
  \newblock A problem we should try to solve before the ISPN '43 deadline,
  \newblock \emph{Letter to Leonhard Euler}, 1742.
\end{thebibliography}
