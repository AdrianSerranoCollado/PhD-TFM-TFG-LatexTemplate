%%%%%%%%%%%%%%%%%%%%%%%%%%%%%%%%%%%%%%%%%%%%%%%%%%%%%%%%%%%%%%%%%%%%%%%%%%%
%
% Generic template for TFC/TFM/TFG/Tesis
%
% $Id: manual.tex,v 1.14 2016/03/31 10:44:12 macias Exp $
%
% By:
%  + Javier Macías-Guarasa. 
%    Departamento de Electrónica
%    Universidad de Alcalá
%  + Roberto Barra-Chicote. 
%    Departamento de Ingeniería Electrónica
%    Universidad Politécnica de Madrid   
% 
% Based on original sources by Roberto Barra, Manuel Ocaña, Jesús Nuevo,
% Pedro Revenga, Fernando Herránz and Noelia Hernández. Thanks a lot to
% all of them, and to the many anonymous contributors found (thanks to
% google) that provided help in setting all this up.
%
% See also the additionalContributors.txt file to check the name of
% additional contributors to this work.
%
% If you think you can add pieces of relevant/useful examples,
% improvements, please contact us at (macias@depeca.uah.es)
%
% You can freely use this template and please contribute with
% comments or suggestions!!!
%
%%%%%%%%%%%%%%%%%%%%%%%%%%%%%%%%%%%%%%%%%%%%%%%%%%%%%%%%%%%%%%%%%%%%%%%%%%%


\chapter{Manual de usuario}
\label{cha:manual-de-usuario}

\section{Introducción}
\label{sec:intro-manual-de-usuario}

Blah, blah, blah\ldots


\section{Manual}
\label{sec:sec-manual-de-usuario}

Pues eso.


\section{Ejemplos de inclusión de fragmentos de código fuente}
\label{sec:codigo-fuente}

Para la inclusión de código fuente se utiliza el paquete
\texttt{listings}, para el que se han definido algunos estilos de
ejemplo que pueden verse en el fichero \texttt{Config/preamble.tex} y
que se usan a continuación.

Así se inserta código fuente, usando el estilo \texttt{CppExample} que
hemos definido en el preamble, escribiendo el código directamente :

% OJO: caracteres especiales no funcionan en utf8. Alternativa mas abajo
% cambiando el inputencoding y usando un fichero externo iso-8859-1
\begin{lstlisting}[style=CppExample]
#include <stdio.h>

// Esto es una funcion de prueba
void funcionPrueba(int argumento)
{	
	int prueba = 1;

  printf("Esto es una prueba [%d][%d]\n", argumento, prueba);

}
\end{lstlisting}

O bien insertando directamente código de un fichero externo, como en el
ejemplo \ref{cod:sample1}, usando
\texttt{\textbackslash{}lstinputlisting} y cambiando el estilo a
\texttt{Cbluebox} (además de usar el entorno \texttt{codefloat} para
evitar pagebreaks, etc.).

\begin{codefloat}
\inputencoding{latin1}
\lstinputlisting[style=Cbluebox]{appendix/function.c}
\inputencoding{utf8}
\caption{Ejemplo de código fuente con un \texttt{lstinputlisting} dentro
de un \texttt{codefloat}}
\label{cod:sample1}
\end{codefloat}


O por ejemplo en matlab, definiendo settings en lugar de usar estilos
definidos:

\lstset{language=matlab}
\lstset{tabsize=2}
\lstset{commentstyle=\textit}
\lstset{stringstyle=\ttfamily, basicstyle=\small}
\begin{lstlisting}[frame=trbl]{}
%
% add_simple.m - Simple matlab script to run with condor
%
a = 9;
b = 10;

c = a+b;

fprintf(1, 'La suma de %d y %d es igual a %d\n', a, b, c);
\end{lstlisting}

O incluso como en el listado \ref{cod:sample2}, usando un layout más refinado (con
los settings de \url{http://www.rafalinux.com/?p=599} en un \texttt{lststyle}
\texttt{Cnice}).


\begin{codefloat}
\lstinputlisting[style=Cnice]{appendix/hello.c}
\caption{Ejemplo de código fuente con estilo \texttt{Cnice}, de nuevo
  con un \texttt{lstinputlisting} dentro de un \texttt{codefloat}}
\label{cod:sample2}
\end{codefloat}

Y podemos reutilizar estilos cambiando algún parámetro, como podemos ver
en el listado \ref{cod:sample3}, en el que hemos vuelto a usar el estilo
\texttt{Cnice} eliminando la numeración.


\begin{codefloat}
\lstinputlisting[style=Cnice,numbers=none]{appendix/hello.c}
\caption{Ejemplo de código fuente con estilo \texttt{Cnice}, modificado
para que no aparezca la numeración.}
\label{cod:sample3}
\end{codefloat}


\noindent
Ahora compila usando \texttt{gcc}:


\begin{lstlisting}[style=console, numbers=none]
$ gcc  -o hello hello.c
\end{lstlisting}

Y también podemos poner ejemplos de código \textit{coloreado}, como se
muestra en el \ref{cod:sample5}.

\begin{codefloat}
\lstinputlisting[style=Ccolor]{appendix/hello.c}
\caption{Ejemplo con colores usando el estilo \texttt{Ccolor}}
\label{cod:sample5}
\end{codefloat}

Finalmente aquí tenéis un ejemplo de código shell, usando el estilo
\texttt{BashInputStyle}:

\begin{lstlisting}[style=BashInputStyle, numbers=none]
#!/bin/sh

HOSTS_ALL="gc000 gc001 gc002 gc003 gc004 gc005 gc006 gc007"

for h in $HOSTS_ALL
do
	echo "Running [$*] in $h..."
  echo -n "   "
  ssh root@$h $*
done
\end{lstlisting}


\section{Ejemplos de inclusión de algoritmos}
\label{sec:algoritmos}

Desde la versión de abril de 2014, empezamos a usar el paquete
\texttt{algorithm2e} para incluir algoritmos, y hay ajustes específicos
y dependientes de este paquete tanto en \texttt{Config/preamble.tex}
como en \texttt{cover/extralistings.tex} (editadlos según vuestras
necesidades). 

Hay otras opciones disponibles (por ejemplo las descritas en
\url{http://en.wikibooks.org/wiki/LaTeX/Algorithm}), y podemos
abordarlas, pero por el momento nos quedamos con \texttt{algorithm2e}.

Incluimos dos ejemplos directamente del manual: uno sencillo en el
algoritmo~\ref{alg:howto}, y otro un poco más complicado en el
algoritmo~\ref{alg:restriction}.

\begin{algorithm}[H]
 \caption{How to write algorithms}
 \label{alg:howto}
 \KwData{this text}
 \KwResult{how to write algorithm with \LaTeX2e }
 initialization\;
 \While{not at end of this document}{
  read current\;
  \eIf{understand}{
   go to next section\;
   current section becomes this one\;
   }{
   go back to the beginning of current section\;
  }
 }
\end{algorithm}


\begin{algorithm}
  \caption{IntervalRestriction\label{IR}}
  \label{alg:restriction}
  \DontPrintSemicolon
  % \dontprintsemicolon
  \KwData{$G=(X,U)$ such that $G^{tc}$ is an order.}
  \KwResult{$G'=(X,V)$ with $V\subseteq U$ such that $G'^{tc}$ is an
    interval order.}
  \Begin{
    $V \longleftarrow U$\;
    $S \longleftarrow \emptyset$\;
    \For{$x\in X$}{
      $NbSuccInS(x) \longleftarrow 0$\;
      $NbPredInMin(x) \longleftarrow 0$\;
      $NbPredNotInMin(x) \longleftarrow |ImPred(x)|$\;
    }
    \For{$x \in X$}{
      \If{$NbPredInMin(x) = 0$ {\bf and} $NbPredNotInMin(x) = 0$}{
        $AppendToMin(x)$}
    }
    \nl\While{$S \neq \emptyset$}{\label{InRes1}
      \nlset{REM} remove $x$ from the list of $T$ of maximal index\;\label{InResR}
      \lnl{InRes2}\While{$|S \cap ImSucc(x)| \neq |S|$}{
        \For{$ y \in S-ImSucc(x)$}{
          \{ remove from $V$ all the arcs $zy$ : \}\;
          \For{$z \in ImPred(y) \cap Min$}{
            remove the arc $zy$ from $V$\;
            $NbSuccInS(z) \longleftarrow NbSuccInS(z) - 1$\;
            move $z$ in $T$ to the list preceding its present list\;
            \{i.e. If $z \in T[k]$, move $z$ from $T[k]$ to
            $T[k-1]$\}\;
          }
          $NbPredInMin(y) \longleftarrow 0$\;
          $NbPredNotInMin(y) \longleftarrow 0$\;
          $S \longleftarrow S - \{y\}$\;
          $AppendToMin(y)$\;
        }
      }
      $RemoveFromMin(x)$\;
    }
  }
\end{algorithm}

%%% Local Variables:
%%% TeX-master: "../book"
%%% End:


