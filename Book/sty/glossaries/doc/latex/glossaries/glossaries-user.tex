\documentclass[report,inlinetitle]{nlctdoc}

\usepackage[inner=0.5in,includemp]{geometry}

\usepackage{array}
\usepackage{alltt}
\usepackage{pifont}
\ifpdf
 \usepackage{lmodern}
 \usepackage{mathpazo}
 \usepackage[scaled=.88]{helvet}
 \usepackage{courier}
 \usepackage{metalogo}
\else
 \newcommand{\grimace}{{\fontencoding {U}\fontfamily{futs}\selectfont \char77}}
 \providecommand{\XeLaTeX}{XeLaTeX}
 \providecommand{\XeTeX}{XeTeX}
\fi
\usepackage[colorlinks,
            bookmarks,
            hyperindex=false,
            pdfauthor={Nicola L.C. Talbot},
            pdftitle={User Manual for glossaries.sty},
            pdfkeywords={LaTeX,package,glossary,acronyms}]{hyperref}
\usepackage{xr-hyper}
\usepackage[
  xindy,
  nonumberlist,
  toc,
  nopostdot,
  nogroupskip,
  style=altlist
]{glossaries}
\usepackage{glossary-inline}

\pagestyle{headings}

\glsnoexpandfields

\renewcommand*{\glsseeformat}[3][\seename]{%
  (\xmakefirstuc{#1} \glsseelist{#2}.)%
}

\renewcommand*{\glossarypreamble}{%
\emph{This glossary style was setup using:}
\begin{ttfamily}
\begin{tabbing}
\cs{usepackage}[\=xindy,\\
\+\>nonumberlist,\\
  toc,\\
  nopostdot,\\
  style=altlist,\\
  nogroupskip]\{glossaries\}
\end{tabbing}
\end{ttfamily}
}

\ifpdf
\else
  % Need an extra line break in the html version
  % (Don't have time to fiddle with cfg files!)
  \renewcommand*{\glossentry}[2]{%
    \item[\glsentryitem{#1}\glstarget{#1}{\glossentryname{#1}}]\mbox{}\newline
      \glossentrydesc{#1}\glspostdescription\space #2\newline}%
\fi

\makeglossaries

\renewcommand*{\main}[1]{\hyperpage{#1}}
\newcommand*{\htextbf}[1]{\textbf{\hyperpage{#1}}}
\newcommand*{\itermdef}[1]{\index{#1|htextbf}}

\newcommand{\itempar}[1]{\item[{#1}]\mbox{}\par}

\newcommand{\glslike}{\hyperref[sec:gls-like]{\cs{gls}-like}}
\newcommand{\glstextlike}{\hyperref[sec:glstext-like]{\cs{glstext}-like}}

\IndexPrologue{\chapter*{\indexname}
 \markboth{\indexname}{\indexname}}

\longnewglossaryentry{indexingapp}%
{
  name={Indexing application},
  text={indexing application}
}%
{%
  An application (piece of software) separate from
  \TeX/\LaTeX\ that collates and sorts information that has an
  associated page reference. Generally the information is an index
  entry but in this case the information is a glossary entry.
  There are two main indexing applications that are used with \TeX:
  \gls{makeindex} and \gls{xindy}. These are both 
  \gls{cli} applications.
}

\longnewglossaryentry{cli}
{
  name={Command Line Interface (CLI)},
  first={command line interface (CLI)},
  text={CLI}
}
{%
  An application that doesn't have a graphical user
  interface. That is, an application that doesn't have any windows,
  buttons or menus and can be run in 
  \href{http://www.dickimaw-books.com/latex/novices/html/terminal.html}{a command
  prompt or terminal}.
}

\longnewglossaryentry{xindy}{%
  name={\appfmt{xindy}\index{xindy=\appfmt{xindy}|htextbf}},
  sort={xindy},
  text={\appfmt{xindy}\iapp{xindy}}
}%
{%
  A flexible \gls{indexingapp} with multilingual
  support written in Perl.
}

\newglossaryentry{makeindex}{%
name={\appfmt{makeindex}\index{makeindex=\appfmt{makeindex}|htextbf}},%
sort={makeindex},%
text={\appfmt{makeindex}\iapp{makeindex}},%
description={An \gls{indexingapp}.},
}

\newglossaryentry{makeglossaries}{%
name={\appfmt{makeglossaries}\index{makeglossaries=\appfmt{makeglossaries}|htextbf}},%
sort={makeglossaries},%
text={\appfmt{makeglossaries}\iapp{makeglossaries}},%
description={A custom designed Perl script interface 
to \gls{xindy} and \gls{makeindex} provided with the \styfmt{glossaries}
package.}
}

\longnewglossaryentry{makeglossariesgui}{%
name={\appfmt{makeglossariesgui}\index{makeglossariesgui=\appfmt{makeglossariesgui}|htextbf}},%
sort={makeglossariesgui},%
text={\appfmt{makeglossariesgui}\iapp{makeglossariesgui}}%
}%
{%
  A Java GUI alternative to \gls{makeglossaries} that
  also provides diagnostic tools. Available separately on CTAN.
}

\newglossaryentry{numberlist}{%
name={Number list\itermdef{number list}},%
sort={number list},%
text={number list\iterm{number list}},%
description={A list of \glslink{entrylocation}{entry locations} (also 
called a location list). The number list can be suppressed using the 
\pkgopt{nonumberlist} package option.}
}

\newglossaryentry{entrylocation}{%
name={Entry location\itermdef{entry location}},%
sort={entry location},%
text={entry location\iterm{entry location}},%
description={The location of the entry in the document. This
defaults to the page number on which the entry appears. An entry may
have multiple locations.}
}

\newglossaryentry{locationlist}{%
name={Location list},%
text={location list},
sort={location list},%
description={A list of \glslink{entrylocation}{entry locations}.
See \gls{numberlist}.}%
}

\newglossaryentry{linktext}{%
name={Link text\itermdef{link text}},
sort={link text},%
text={link text\iterm{link text}},
description={The text produced by commands such as \ics{gls}. It may
or may not be a hyperlink to the glossary.}
}

\let\glsd\glsuseri
\let\glsation\glsuserii

\longnewglossaryentry{firstuse}{%
name={First use\ifirstuse},
sort={first use},%
text={first use},%
user1={first used}}
{%
  The first time a glossary entry is used (from the start of the 
  document or after a reset) with one of the
  following commands: \ics{gls}, \ics{Gls}, \ics{GLS}, \ics{glspl},
  \ics{Glspl}, \ics{GLSpl} or \ics{glsdisp}.
  \glsseeformat{firstuseflag,firstusetext}{}
}

\longnewglossaryentry{firstuseflag}{%
name={First use flag\ifirstuseflag},
sort={first use flag},%
text={first use flag}%
}
{%
  A conditional that determines whether or not the entry
  has been used according to the rules of \gls{firstuse}. Commands
  to unset or reset this conditional are described in 
  \sectionref{sec:glsunset}.
}

\newglossaryentry{firstusetext}{%
name={First use text\ifirstusetext},
sort={first use text},%
text={first use text},%
description={The text that is displayed on \gls{firstuse}, which is
governed by the \gloskey{first} and \gloskey{firstplural} keys of
\ics{newglossaryentry}. (May be overridden by
\ics{glsdisp} or by \ics{defglsentry}.)}%
}

\longnewglossaryentry{sanitize}{%
name={Sanitize\itermdef{sanitize}},%
sort={sanitize},
text={sanitize\iterm{sanitize}},%
user1={sanitized\protect\iterm{sanitize}},%
user2={sanitization\protect\iterm{sanitize}}%
}%
{%
  Converts command names into character sequences. That is, a command called,
  say, \cs{foo}, is converted into the sequence of characters:
  \cs{}, \texttt{f}, \texttt{o}, \texttt{o}. Depending on the font,
  the backslash character may appear as a dash when used in the
  main document text, so \cs{foo} will appear as: ---foo.

  Earlier versions of \styfmt{glossaries} used this technique to write
  information to the files used by the indexing applications to
  prevent problems caused by fragile commands. Now, this is only used
  for the \gloskey{sort} key.
}

\newglossaryentry{latinchar}{%
  name={Latin Character\itermdef{Latin character}},
  text={Latin character\iterm{Latin character}},
  sort={Latin character},
  description={One of the letters \texttt{a}, \ldots, \texttt{z}, 
  \texttt{A}, \ldots, \texttt{Z}\@.
  See also \gls{exlatinchar}.}
}

\newglossaryentry{exlatinchar}{%
  name={Extended Latin Character\itermdef{extended Latin character}},
  text={extended Latin character\iterm{extended Latin character}},
  sort={extended Latin character},
  description={A character that's created by combining \glspl{latinchar}
  to form ligatures (e.g.\ \ae) or by applying diacritical marks
  to a~\gls*{latinchar} or characters (e.g.\ \'a or \o). 
  See also \gls{nonlatinchar}.}
}

\newglossaryentry{latexexlatinchar}{%
  name={Standard \LaTeX\ Extended Latin Character\itermdef{standard
\LaTeX\ extended Latin character}},
  text={standard \LaTeX\ extended Latin character\iterm{standard
\LaTeX\ extended Latin character}},
  sort={standard LaTeX extended Latin character},
  description={An \gls{exlatinchar} that can be created by a~core
\LaTeX\ command, such as \cs{o} (\o) or \cs{'}\texttt{e} (\'e). 
  That is, the character can be produced without the need to load 
  a~particular package.}
}

\newglossaryentry{nonlatinchar}{%
  name={Non-Latin Character\itermdef{non-Latin character}},
  text={non-Latin character\iterm{non-Latin character}},
  sort={non-Latin character},
  description={An \gls{exlatinchar} or a~character that isn't
  a~\gls{latinchar}.}
}

\newglossaryentry{latinalph}{%
  name={Latin Alphabet\itermdef{Latin alphabet}},
  text={Latin alphabet\iterm{Latin alphabet}},
  sort={Latin alphabet},
  description={The alphabet consisting of \glspl{latinchar}.
  See also \gls{exlatinalph}.}
}

\newglossaryentry{exlatinalph}{%
  name={Extended Latin Alphabet\itermdef{Extended Latin Alphabet}},
  text={extended Latin alphabet},
  sort={extended Latin alphabet},
  description={An alphabet consisting of \glspl{latinchar}
  and \glspl{exlatinchar}.}
}

\newglossaryentry{nonlatinalph}{%
  name={Non-Latin Alphabet\itermdef{Non-Latin Alphabet}},
  text={non-Latin alphabet},
  sort={non-Latin alphabet},
  description={An alphabet consisting of \glspl{nonlatinchar}.}
}

\ifpdf
\externaldocument{glossaries-code}
\fi

\doxitem{Option}{option}{package options}
\doxitem{GlsKey}{key}{glossary keys}
\doxitem{Style}{style}{glossary styles}
\doxitem{Counter}{counter}{glossary counters}

\setcounter{IndexColumns}{2}

\newcommand*{\tick}{\ding{51}}

\newcommand*{\yes}{\ding{52}}
\newcommand*{\no}{\ding{56}}

\makeatletter
\newcommand*{\optionlabel}[1]{%
 \@glstarget{option#1}{}\textbf{Option~#1}}
\makeatother

\newcommand*{\opt}[1]{\hyperlink{option#1}{Option~#1}}
\newcommand*{\optsor}[2]{Options~\hyperlink{option#1}{#1}
or~\hyperlink{option#2}{#2}}
\newcommand*{\optsand}[2]{Options~\hyperlink{option#1}{#1}
and~\hyperlink{option#2}{#2}}


\newcommand*{\ifirstuse}{\iterm{first use}}
\newcommand*{\ifirstuseflag}{\iterm{first use>flag}}
\newcommand*{\ifirstusetext}{\iterm{first use>text}}

\newcommand*{\firstuse}{\gls{firstuse}}
\newcommand*{\firstuseflag}{\gls{firstuseflag}}
\newcommand*{\firstusetext}{\gls{firstusetext}}


\newcommand*{\istkey}[1]{\appfmt{#1}\index{makeindex=\appfmt{makeindex}>#1=\texttt{#1}|hyperpage}}
\newcommand*{\locfmt}[1]{\texttt{#1}\SpecialMainIndex{#1}}
\newcommand*{\mkidxspch}{\index{makeindex=\appfmt{makeindex}>special characters|hyperpage}}

\newcommand*{\igloskey}[2][newglossaryentry]{\icsopt{#1}{#2}}
\newcommand*{\gloskey}[2][newglossaryentry]{\csopt{#1}{#2}}

\newcommand*{\glostyle}[1]{\textsf{#1}\index{glossary styles:>#1={\protect\ttfamily#1}|main}}

\newcommand*{\acrstyle}[1]{\textsf{#1}\index{acronym styles:>#1={\protect\ttfamily#1}|main}}

\newcounter{sample}
\newcommand{\exitem}[2][sample]{%
  \item[\texttt{#1#2.tex}]%
  \refstepcounter{sample}\label{ex:#1#2}}

\newenvironment{samplelist}%
{\begin{description}}%
{\end{description}}

\newcommand*{\samplefile}[2][sample]{%
  \hyperref[ex:#1#2]{\texttt{#1#2.tex}}}

\ifpdf
  \newcommand*{\htmldoc}[2]{\qt{#1}}
\else
  \newcommand*{\htmldoc}[2]{\href{#2.html}{\qt{#1}}}
\fi

\begin{document}
\MakeShortVerb{"}
\DeleteShortVerb{\|}

 \title{User Manual for glossaries.sty v4.11}
 \author{Nicola L.C. Talbot\\%
  \url{http://www.dickimaw-books.com/}}

 \date{2014-09-01}
 \maketitle

\begin{abstract}
The \styfmt{glossaries} package provides a means to define terms or
acronyms or symbols that can be referenced within your document.
Sorted lists with collated \glslink{entrylocation}{locations} can be 
generated either using \TeX\ or using a supplementary \gls{indexingapp}.
\end{abstract}

\begin{important}
Documents have various styles when it comes to presenting glossaries
or lists of terms or notation. People have their own preferences and
to a large extent this is determined by the kind of information that
needs to go in the glossary. They may just have symbols with
terse descriptions or they may have long technical words with
complicated descriptions. The \styfmt{glossaries} package is
flexible enough to accommodate such varied requirements, but this
flexibility comes at a price: a~big manual.

\aargh\ If you're freaking out at the size of this manual, start with
\texttt{glossariesbegin.pdf} (\qt{The glossaries package: a guide
for beginnners}). You should find it in the same directory as this
document or try \texttt{texdoc glossariesbegin.pdf}. Once you've got
to grips with the basics, then come back to this manual to find out
how to adjust the settings.
\end{important}

\noindent
The \styfmt{glossaries} bundle comes with the following documentation:
\begin{description}
\item[\url{glossariesbegin.pdf}] 
If you are a complete beginner, start with 
\htmldoc{The glossaries package: a guide for
beginners}{glossariesbegin}.

\item[\url{glossary2glossaries.pdf}] 
If you are moving over from the obsolete \sty{glossary} package,
read \htmldoc{Upgrading from the glossary package to the
glossaries package}{glossary2glossaries}.

\item[glossaries-user.pdf]
This document is the main user guide for the \styfmt{glossaries}
package.

\item[\url{mfirstuc-manual.pdf}]
The commands provided by the \sty{mfirstuc} package are briefly
described in \htmldoc{mfirstuc.sty: uppercasing first
letter}{mfirstuc-manual}.

\item[\url{glossaries-code.pdf}]
Advanced users wishing to know more about the inner workings of all the
packages provided in the \styfmt{glossaries} bundle should read
\qt{Documented Code for glossaries v4.11}.
This includes the documented code for the \sty{mfirstuc} package.

\item[INSTALL] Installation instructions.

\item[CHANGES] Change log.

\item[README] Package summary.

\end{description}

\begin{important}
If you use \sty{hyperref} and \styfmt{glossaries}, you must load
\sty{hyperref} \emph{first}. Similarly the \sty{doc} package must
also be loaded before \styfmt{glossaries}. (If \sty{doc} is loaded,
the file extensions for the default main glossary are changed to
\filetype{gls2}, \filetype{glo2} and \filetype{.glg2} to avoid
conflict with \sty{doc}'s changes
glossary.)\hypertarget{pdflatexnote}{}%

If you are using \sty{hyperref}, it's best to use \app{pdflatex}
rather than \app{latex} (DVI format) as \appfmt{pdflatex} deals with
hyperlinks much better. If you use the DVI format, you will
encounter problems where you have long hyperlinks or hyperlinks in
subscripts or superscripts. This is an issue with the DVI format not
with \styfmt{glossaries}.
\end{important}

Other documents that describe using the \styfmt{glossaries} package include:
\href{http://www.dickimaw-books.com/latex/thesis/}{Using LaTeX to Write a PhD Thesis}
and
\href{http://www.latex-community.org/know-how/latex/55-latex-general/263-glossaries-nomenclature-lists-of-symbols-and-acronyms}{Glossaries,
Nomenclature, Lists of Symbols and Acronyms}.

\clearpage
\pdfbookmark[0]{Contents}{contents}
\tableofcontents
\clearpage
\pdfbookmark[0]{List of Examples}{examples}
\listofexamples
\clearpage
\pdfbookmark[0]{List of Tables}{tables}
\listoftables

\clearpage
\printglossaries

\glsresetall

 \chapter{Introduction}
\label{sec:intro}

The \styfmt{glossaries} package is provided to assist generating
lists of terms, symbols or abbreviations (glossaries). It has a certain amount of flexibility, allowing the
user to customize the format of the glossary and define multiple
glossaries. It also supports glossary styles that
include symbols (in addition to a name and description) for glossary
entries. There is provision for loading a database of glossary
terms. Only those terms used\footnote{That is, if the term has been
referenced using any of the commands described in
\sectionref{sec:glslink} and \sectionref{sec:glsadd} or via
\ics{glssee} (or the \gloskey{see} key) or commands such as
\ics{acrshort}.}\ in the document will be added to the glossary.

\textbf{This package replaces the \sty{glossary} package which is
now obsolete.} Please see the document \qtdocref{Upgrading from the
glossary package to the glossaries package}{glossary2glossaries} for
assistance in upgrading.

One of the strengths of this package is its flexibility, however
the drawback of this is the necessity of having a large manual
that can cover all the various settings. If you are daunted by the
size of the manual, try starting off with the much shorter
\docref{guide for beginners}{glossariesbegin}.

\begin{important}
There's a~common misconception that you have to have Perl installed
in order to use the \styfmt{glossaries} package. Perl is \emph{not}
a~requirement but it does increase the available options,
particularly if you use an \gls{exlatinalph} or a~\gls{nonlatinalph}.
\end{important}

The basic idea behind the \styfmt{glossaries} package is that you
first define your entries (terms, symbols or abbreviations). Then
you can reference these within your document (like \cs{cite} or
\cs{ref}).  You can also, optionally, display a~list of the entries
you have referenced in your document (the glossary). This last part,
displaying the glossary, is the part that most new users find
difficult. There are three options:

\begin{description}
\item[]\optionlabel1: 

 This is the simplest option but it's slow and if
 you want a sorted list, it doesn't work well for \glspl{exlatinalph} or 
 \glspl{nonlatinalph}. However, if you use the
 \pkgopt[false]{sanitizesort} package option (the default for
 Option~1) then the \glslink{latexexlatinchar}{standard \LaTeX\ accent commands} will be
 ignored, so if an entry's name is set to \verb|{\'e}lite| then the
 sort will default to \texttt{elite} if 
 \pkgopt[false]{sanitizesort} is used
 and will default to \verb|\'elite| if \pkgopt[true]{sanitizesort}
 is used.

  \begin{enumerate}
    \item Add \cs{makenoidxglossaries} to your preamble (before you
    start defining your entries, as described in
    \sectionref{sec:newglosentry}).

    \item Put
\begin{definition}
\cs{printnoidxglossary}
\end{definition}
    where you want your list of entries to appear (described in
    \sectionref{sec:printglossary}).

    \item Run \LaTeX\ twice on your document. (As you would do to
    make a~table of contents appear.) For example, click twice on
    the \qt{typeset} or \qt{build} or \qt{PDF\LaTeX} button in your editor.
  \end{enumerate}

\item\optionlabel2:

   This option uses a~\gls{cli} application called \gls{makeindex} to sort 
   the entries. This application comes with all modern \TeX\ distributions, 
   but it's hard-coded for the non-extended \gls{latinalph}, so 
   it doesn't work well for \glspl{exlatinalph} or
   \glspl{nonlatinalph}. This process involves making \LaTeX\ write the 
   glossary information to a temporary file which \gls{makeindex} reads. 
   Then \gls{makeindex} writes a~new file containing the code to typeset 
   the glossary. \LaTeX\ then reads this file in on the next run.

   \begin{enumerate}
    \item Add \cs{makeglossaries} to your preamble (before you start
    defining your entries, as described in
    \sectionref{sec:newglosentry}).

    \item Put
\begin{definition}
\cs{printglossary}
\end{definition}
    where you want your list of entries to appear (described in
    \sectionref{sec:printglossary}).

    \item Run \LaTeX\ on your document. This creates files with the
    extensions \texttt{.glo} and \texttt{.ist} (for example, if your 
    \LaTeX\ document is called \texttt{myDoc.tex}, then you'll have 
    two extra files called \texttt{myDoc.glo} and \texttt{myDoc.ist}).
    If you look at your document at this point, you won't see the 
    glossary as it hasn't been created yet.

    \item Run \gls{makeindex} with the \texttt{.glo} file as the
    input file and the \texttt{.ist} file as the style so that
    it creates an output file with the extension \texttt{.gls}. If
    you have access to a terminal or a command prompt (for example, the
    MSDOS command prompt for Windows users or the bash console for
    Unix-like users) then you need to run the command:
\begin{verbatim}
makeindex -s myDoc.ist -o myDoc.gls myDoc.glo
\end{verbatim}
   (Replace \texttt{myDoc} with the base name of your \LaTeX\
    document file. Avoid spaces in the file name.) If you don't know
    how to use the command prompt, then you can probably access
    \gls{makeindex} via your text editor, but each editor has a
    different method of doing this, so I~can't give a~general
    description. You will have to check your editor's manual.

    The default sort is word order (\qt{sea lion} comes before
\qt{seal}). 
    If you want letter ordering you need to add the \texttt{-l}
    switch:
\begin{verbatim}
makeindex -l -s myDoc.ist -o myDoc.gls myDoc.glo
\end{verbatim}
    (See \sectionref{sec:makeindexapp} for further details on using 
    \gls*{makeindex} explicitly.)

    \item Once you have successfully completed the previous step,
    you can now run \LaTeX\ on your document again.
   \end{enumerate}

   This is the default option (although you still need to use
   \cs{makeglossaries} to ensure the glossary files are created).

\item\optionlabel3:

   This option uses a~\gls{cli} application called
   \gls{xindy} to sort the entries. This application is more flexible than 
   \texttt{makeindex} and is able to sort \glspl{exlatinalph} or 
   \glspl{nonlatinalph}. The \gls{xindy} application comes with \TeX~Live 
   but not with MiK\TeX.  Since \gls{xindy} is a Perl script, if you are
   using MiK\TeX\ you will not only need to install \gls{xindy}, you
   will also need to install Perl. In a~similar way to \opt2, this 
   option involves making \LaTeX\ write the glossary information to 
   a~temporary file which \gls{xindy} reads. Then \gls{xindy} 
   writes a~new file containing the code to typeset the glossary. 
   \LaTeX\ then reads this file in on the next run.

   \begin{enumerate}
     \item Add the \pkgopt{xindy} option to the \styfmt{glossaries}
package option list:
\begin{verbatim}
\usepackage[xindy]{glossaries}
\end{verbatim}

     \item Add \cs{makeglossaries} to your preamble (before you start
     defining your entries, as described in \sectionref{sec:newglosentry}).

    \item Run \LaTeX\ on your document. This creates files with the
    extensions \texttt{.glo} and \texttt{.xdy} (for example, if your 
    \LaTeX\ document is called \texttt{myDoc.tex}, then you'll have 
    two extra files called \texttt{myDoc.glo} and \texttt{myDoc.xdy}).
    If you look at your document at this point, you won't see the 
    glossary as it hasn't been created yet.

    \item Run \gls{xindy} with the \texttt{.glo} file as the
    input file and the \texttt{.xdy} file as a~module so that
    it creates an output file with the extension \texttt{.gls}. You 
    also need to set the language name and input encoding. If
    you have access to a terminal or a command prompt (for example, the
    MSDOS command prompt for Windows users or the bash console for
    Unix-like users) then you need to run the command (all on one
    line):
\begin{verbatim}
xindy  -L english -C utf8 -I xindy -M myDoc 
-t myDoc.glg -o myDoc.gls myDoc.glo
\end{verbatim}
    (Replace \texttt{myDoc} with the base name of your \LaTeX\
    document file. Avoid spaces in the file name. If necessary, also replace
    \texttt{english} with the name of your language and \texttt{utf8}
    with your input encoding.) If you don't know
    how to use the command prompt, then you can probably access
    \gls{xindy} via your text editor, but each editor has a
    different method of doing this, so I~can't give a~general
    description. You will have to check your editor's manual.

    The default sort is word order (\qt{sea lion} comes before
\qt{seal}). 
    If you want letter ordering you need to add the
    \pkgopt[letter]{order} package option:
\begin{verbatim}
\usepackage[xindy,order=letter]{glossaries}
\end{verbatim}
    (See \sectionref{sec:xindyapp} for further details on using 
    \gls*{xindy} explicitly.)

    \item Once you have successfully completed the previous step,
    you can now run \LaTeX\ on your document again.

   \end{enumerate}

\end{description}

For \optsand23, it can be difficult to remember all the
parameters required for \gls{makeindex} or \gls{xindy}, so the
\styfmt{glossaries} package provides a~script called
\gls{makeglossaries} that reads the \texttt{.aux} file to
determine what settings you have used and will then run
\gls{makeindex} or \gls{xindy}. Again, this is a~command line
application and can be run in a~terminal or command prompt. For
example, if your \LaTeX\ document is in the file \texttt{myDoc.tex},
then run:
\begin{verbatim}
makeglossaries myDoc
\end{verbatim}
(Replace \texttt{myDoc} with the base name of your \LaTeX\ document
file. Avoid spaces in the file name.) This is described in more
detail in \sectionref{sec:makeglossaries}.

\begin{important}
The \texttt{.gls} and \texttt{.glo} are temporary files
created to help build your document. You should not edit or explicitly input
them. However, you may need to delete them if something goes wrong
and you need to do a fresh build.
\end{important}

An overview of these three options is given in
\tableref{tab:options}.

\begin{table}[htbp]
 \caption{Glossary Options: Pros and Cons}
 \label{tab:options}
 {%
 \centering
 \begin{tabular}{>{\raggedright}p{0.35\textwidth}ccc}
   & \bfseries \opt1 & \bfseries \opt2 & \bfseries \opt3\\
   Requires an external application? &
   \no & \yes & \yes\\
   Requires Perl? &
   \no & \no & \yes\\
   Can sort \glspl{exlatinalph}
   or \glspl{nonlatinalph}? &
   \no\textsuperscript{\textdagger} & \no & \yes\\
   Efficient sort algorithm? &
   \no & \yes & \yes\\
   Can use a different sort method for each glossary? &
   \yes & \no & \no\\
   Can form ranges in the location lists? &
   \no & \yes & \yes\\
   Can have non-standard locations in the location lists? &
   \yes & \no & \yes\\
   Maximum hierarchical depth &
   Unlimited & 3 & Unlimited\\
   \ics{glsdisplaynumberlist} reliable? &
   \yes & \no & \no\\
   \ics{newglossaryentry} restricted to preamble? &
   \yes & \no & \no\\
   Requires additional write registers? &
   \no & \yes & \yes\\
   Default value of \pkgopt{sanitizesort} package option &
   \texttt{false} & \texttt{true} & \texttt{true}
 \end{tabular}
 \par
 }\textsuperscript{\textdagger} Strips standard \LaTeX\ accents
(that is, accents generated by core \LaTeX\ commands) so,
for example, \ics{AA} is treated the same as A.
\end{table}

This document uses the \styfmt{glossaries} package. For example,
when viewing the PDF version of this document in a
hyperlinked-enabled PDF viewer (such as Adobe Reader or Okular) if
you click on the word \qt{\gls{xindy}} you'll be taken to the entry
in the glossary where there's a brief description of
the term \qt{\gls*{xindy}}.

The remainder of this introductory section covers the following:
\begin{itemize}
\item \sectionref{sec:samples} lists the sample documents provided 
with this package.

\item \sectionref{sec:languages} provides information for users who
wish to write in a language other than English.

\item \sectionref{sec:makeglossaries} describes how to use an
\gls{indexingapp} to create the sorted glossaries for your document
(\optsor23).

\end{itemize}

\section{Sample Documents}
\label{sec:samples}

The \styfmt{glossaries} package is provided with some sample
documents that illustrate the various functions. These should
be located in the \texttt{samples} subdirectory (folder) of the
\styfmt{glossaries} documentation directory. This location varies
according to your operating system and \TeX\ distribution. You
can use \texttt{texdoc} to locate the main glossaries documentation.
For example, in a
\href{http://www.dickimaw-books.com/latex/novices/html/terminal.html}{terminal or command prompt}, type:
\begin{prompt}
texdoc -l glossaries
\end{prompt}
This should display a list of all the files in the glossaries
documentation directory with their full pathnames.

If you can't find the sample files on your computer, they are also available
from your nearest CTAN mirror at
\url{http://mirror.ctan.org/macros/latex/contrib/glossaries/samples/}.

The sample documents are as follows\footnote{Note that although I've written
\texttt{latex} in this section, it's better to use \texttt{pdflatex}, where
possible, for the reasons given \hyperlink{pdflatexnote}{earlier}.}:
\begin{samplelist}
\exitem[minimal]{gls} This document is a
minimal working example. You can test your installation using this
file. To create the complete document you will need to do the
following steps:
  \begin{enumerate}
  \item Run \texttt{minimalgls.tex} through \LaTeX\ either by 
  typing
\begin{prompt}
latex minimalgls
\end{prompt}
  in a terminal or by using the relevant button or menu item in
  your text editor or front-end. This will create the required 
  associated files but you will not see the glossary. If you use 
  PDF\LaTeX\ you will also get warnings about non-existent 
  references that look something like:
\begin{verbatim}
pdfTeX warning (dest): name{glo:aca} has been 
referenced but does not exist, 
replaced by a fixed one
\end{verbatim}
  These warnings may be ignored on the first run.

  If you get a \verb"Missing \begin{document}" error, then 
  it's most likely that your version of \sty{xkeyval} is 
  out of date. Check the log file for a warning of that nature. 
  If this is the case, you will need to update the \styfmt{xkeyval}
  package.

  \item Run \gls{makeglossaries} on the document (\sectionref{sec:makeglossaries}). This can
  be done on a terminal either by typing
\begin{prompt}
makeglossaries minimalgls
\end{prompt}
  or by typing
\begin{prompt}
perl makeglossaries minimalgls
\end{prompt}
  If your system doesn't recognise the command \texttt{perl} then
  it's likely you don't have Perl installed. In which case you
  will need to use \gls{makeindex} directly. You can do this
  in a terminal by typing (all on one line):
\begin{prompt}
makeindex -s minimalgls.ist -t minimalgls.glg -o minimalgls.gls minimalgls.glo
\end{prompt}
  (See \sectionref{sec:makeindexapp} for further details on using 
   \gls*{makeindex} explicitly.)

  Note that if you need to specify the full path and the path
  contains spaces, you will need to delimit the file names with
  the double-quote character.

  \item Run \texttt{minimalgls.tex} through \LaTeX\ again (as step~1)
  \end{enumerate}
You should now have a complete document. The number following
each entry in the glossary is the location number. By default, this 
is the page number where the entry was referenced.

\exitem{-noidx} This document illustrates how to use the
\styfmt{glossaries} package without an external \gls{indexingapp} (\opt1).
To create the complete document, you need to do:
\begin{prompt}
latex sample-noidx
latex sample-noidx
\end{prompt}

\exitem{-noidx-utf8} As the previous example, except that it uses
the \sty{inputenc} package.
To create the complete document, you need to do:
\begin{prompt}
latex sample-noidx-utf8
latex sample-noidx-utf8
\end{prompt}

\exitem{4col} This document illustrates
a four column glossary where the entries have a symbol in addition
to the name and description. To create the complete document, you
need to do:
\begin{prompt}
latex sample4col
makeglossaries sample4col
latex sample4col
\end{prompt}
As before, if you don't have Perl installed, you will need to use
\gls{makeindex} directly instead of using
\gls{makeglossaries}. The vertical gap between entries is the
gap created at the start of each group. This can be suppressed
using the \pkgopt{nogroupskip} package option.

\exitem{Acr} This document has some
sample acronyms. It also adds the glossary to the table of contents,
so an extra run through \LaTeX\ is required to ensure the document
is up to date:
\begin{prompt}
latex sampleAcr
makeglossaries sampleAcr
latex sampleAcr
latex sampleAcr
\end{prompt}

\exitem{AcrDesc} This is similar to
the previous example, except that the acronyms have an associated
description. As with the previous example, the glossary is added to
the table of contents, so an extra run through \LaTeX\ is required:
\begin{prompt}
latex sampleAcrDesc
makeglossaries sampleAcrDesc
latex sampleAcrDesc
latex sampleAcrDesc
\end{prompt}

\exitem{Desc} This is similar to the
previous example, except that it defines the acronyms using
\ics{newglossaryentry} instead of \ics{newacronym}. As with the
previous example, the glossary is added to the table of contents, so
an extra run through \LaTeX\ is required:
\begin{prompt}
latex sampleDesc
makeglossaries sampleDesc
latex sampleDesc
latex sampleDesc
\end{prompt}

\exitem{CustomAcr} This document has some sample acronyms with a
custom acronym style. It also adds the glossary to the table of
contents, so an extra run through \LaTeX\ is required:
\begin{prompt}
late