
\chapter{Pliego de condiciones}
\label{cha:pliego}

A continuación se detallan de forma específica los elementos mas importantes de la plataforma desarrollada, centrándose en sus especificaciones técnicas, y el software utilizado en este proyecto. 

\section{Elementos físicos}

\begin{itemize}
    \item Estructura mecánica
    \begin{itemize}
        \item Dimensiones: 675 x 650 x \SI{1000}{\milli\meter}. 
        \item Radio de las ruedas: \SI{90}{\milli\meter}.
        \item Distancia entre ruedas: \SI{530}{\milli\meter}.
        \item Reductora de engranajes planetarios. Reducción x50.9.
        \item Batería. Dos baterías de plomo ácido de \SI{12}{\volt} y \SI{7.2}{\ampere\hour}.
    \end{itemize}
\end{itemize}

\begin{itemize}
    \item Elementos electrónicos
    \begin{itemize}
        \item Tarjeta microcontroladora LPC2129 CAN QuickStart.
        \item Motores PG42775 RS-775244500.
        \item Encoders ME-775.
        \item Módulos driver LMD18200.
        \item Joystick analógico inductivo APEM.
        \item Mando para joystick, con lector de tensión para batería.
        \item Material electrónico (resistencias, condensadores, transistores, etc.) indicados en los esquemáticos.
    \end{itemize}
\end{itemize}

\begin{itemize}
    \item Portátil Lenovo Yoga, para depuración y pruebas.
\end{itemize}

\section{Software}

\begin{itemize}
    \item Procesamiento y análisis de datos: Matlab, versión 2019a, toolboxes Sisotool, Simulink y ROS.
    \item  Paquetes de diseño de circuitos electrónicos: EasyEDA y OrCAD.
    \item Diagramas de flujo realizados con: \href{https://app.diagrams.net/}{https://app.diagrams.net/}.
    \item Procesador de textos: OverLeaf (Latex).
    \item Sistema Operativo: Ubuntu 16.04 LTS.
    \item Versión de ROS: Kinetic.
\end{itemize}

\chapter{Presupuesto}
\label{cha:presupuesto}

En este capítulo se incluye una estimación del coste total para la realización del proyecto. En las siguientes secciones los costes son divididos en función de su origen, exponiendo el subtotal en cada uno de los apartados y al final se añade el total de estas cifras.

\section{Recursos Hardware}
\label{sec:presupuesto-hardware}

Se ha necesitado gran cantidad de recursos hardware para construir la plataforma y que funcione correctamente. Dichos recursos son enumerados en el siguiente listado.

\begin{itemize}
    \item Chasis.
    \item Ruedas.
    \item Baterías.
    \item Motores y encoders.
    \item Piezas 3D.
    \item Material básico de electrónica (resistencias, condensadores, LEDs, etc).
    \item Drivers para motores.
    \item Tarjetas microcontroladoras LPC2129.
    \item Joystick analógico inductivo.
\end{itemize}

El precio y las unidades son incluidos en la tabla \ref{tab:recursos_hardware} junto con el total de los recursos hardware.

\begin{table}[H]
\centering
\begin{tabular}{|c|c|c|c|}
\hline
Concepto                & Precio por Unidad & Cantidad & Subtotal \\ \hline
Chasis                  & 50,00 \euro               & 1        & 50,00 \euro        \\ \hline
Ruedas                  & 10,00 \euro                & 4        & 40,00 \euro       \\ \hline
Baterías                & 40,00 \euro                & 2        & 80,00 \euro       \\ \hline
Motor y encoder         & 20,00 \euro                & 2        & 40,00 \euro       \\ \hline
Piezas 3D               & 15,00 \euro                & 1        & 15,00 \euro       \\ \hline
Material de electrónica & 60,00 \euro                & 1        & 60,00 \euro       \\ \hline
Drivers                 & 30,00 \euro                & 2        & 60,00 \euro       \\ \hline
LPC2129                 & 50,00 \euro                & 2        & 100,00 \euro      \\ \hline
Joystick                & 110,00 \euro               & 1        & 110,00 \euro      \\ \hline
\multicolumn{3}{|c|}{TOTAL}                            & 555,00 \euro     \\ \hline
\end{tabular}
\caption{Recursos hardware usados}
\label{tab:recursos_hardware}
\end{table}

\section{Recursos de desarrollo y pruebas}
\label{sec:presupuesto-software}

Para llevar a cabo este trabajo es necesario el uso de determinado software, que involucra la programación de los nodos, el diseño electrónico de los mismos, la simulación de la plataforma, etc. A continuación se incluye un listado de los programas usados para dichas tareas. El precio se ha calculado en función de la vida útil del recurso (2 años) y el tiempo utilizado (2 meses).

\begin{itemize}
    \item Matlab para diseño de controlador PI, representación de gráficas y pruebas.
    \item OrCAD para diseño de circuitos electrónicos.
    \item $\mu$Vision de Keil para programación de microcontroladores.
    \item FreeCAD. Este programa es gratuito y se ha usado para diseño de piezas 3D y modelado de la plataforma.
\end{itemize}

En la tabla \ref{tab:recursos_software} se incluyen dichos programas con sus respectivos costes de licencia.

\begin{table}[H]
\centering
\begin{tabular}{|c|c|c|c|}
\hline
Concepto    & Precio por Unidad & Coeficiente & Subtotal \\ \hline
Matlab      & 200,00 \euro                 & 0,0833        & 16,66 \euro        \\ \hline
$\mu$Vision & 400,00 \euro                 & 0,0833        & 33,32 \euro        \\ \hline
FreeCAD       & 0,00 \euro                 & 0,0833        & 0,00 \euro       \\ \hline
OrCAD     & 125,00 \euro                 & 0,0833        & 10,41 \euro        \\ \hline
Portatil Lenovo Yoga     & 700,00 \euro                 & 0,0833        & 58,31 \euro        \\ \hline
\multicolumn{3}{|c|}{TOTAL}                & 118,70 \euro        \\ \hline
\end{tabular}
\caption{Recursos de desarrollo y pruebas}
\label{tab:recursos_software}
\end{table}

\section{Recursos Humanos}
\label{sec:presupuesto-mano}

El coste proveniente a los recursos humanos del trabajo provienen de un ingeniero que desarrolla el trabajo completo. A continuación se incluyen dichos costes.

\begin{table}[H]
\centering
\begin{tabular}{|c|c|c|c|}
\hline
Concepto  & Precio por hora & Cantidad de horas & Subtotal \\ \hline
Ingeniero & 50,00 \euro              & 500                 & 25000,00 \euro       \\ \hline
\multicolumn{3}{|c|}{TOTAL}                     & 25000,00 \euro        \\ \hline
\end{tabular}
\caption{Recursos humanos}
\label{tab:recursos_humanos}
\end{table}

\section{Presupuesto de ejecución material}
\label{sec:presupuesto-material}

El presupuesto de ejecución material es la suma de los recursos hardware, software y humanos necesarios para llevar a cabo el trabajo.

\begin{table}[H]
\centering
\begin{tabular}{|c|c|}
\hline
Concepto           & Subtotal \\ \hline
Recursos hardware  & 555,00 \euro        \\ \hline
Recursos software  & 118,70 \euro        \\ \hline
Coste mano de obra & 25000,00 \euro        \\ \hline
TOTAL              & 25673,70 \euro        \\ \hline
\end{tabular}
\caption{Presupuesto de ejecución material}
\label{tab:ejec_material}
\end{table}

\section{Importe de la ejecución por contrata}
\label{sec:presupuesto-ejecucion}

Los costes de la ejecución por contrata deben incluir los gastos derivados del uso de las instalaciones donde se ha llevado a cabo el trabajo, las cargas fiscales, los gastos financiero, las tasas administrativas y las obligaciones de control del proyecto.

Dicho gasto se asume estableciendo un recargo sobre el coste del importe del presupuesto de ejecución material. Dicho recargo equivale al 22\% de dicho importe

\begin{table}[H]
\centering
\begin{tabular}{|c|c|}
\hline
Concepto                                      & Subtotal \\ \hline
22\% del coste total de ejecución de material & 5648,21 \euro       \\ \hline
\end{tabular}
\caption{Importe de ejecución por contrata}
\label{tab:ejec_contrata}
\end{table}


\section{Honorarios facultativos}
\label{sec:presupuesto-facultativos}

Se fija en este proyecto un porcentaje del 7\% sobre el coste total de ejecución por contrata.

\begin{table}[H]
\centering
\begin{tabular}{|c|c|}
\hline
Concepto                                & Subtotal \\ \hline
7\% del coste de ejecución por contrata & 395,38 \euro        \\ \hline
\end{tabular}
\caption{Importe de los honorarios facultativos}
\label{tab:honorarios}
\end{table}

\section{Presupuesto Total}
\label{sec:presupuesto-total}

A continuación, en la tabla \ref{tab:presupuesto_total} se realiza la suma de todos los conceptos del presupuesto tenidos en cuenta en los anteriores apartados.

\begin{table}[H]
\centering
\begin{tabular}{|c|c|}
\hline
Concepto Precio                    & Subtotal \\ \hline
Presupuesto de ejecución material  & 25673,70 \euro        \\ \hline
Importe de la ejecución contratada & 5648,21 \euro        \\ \hline
Horarios facultativos              & 395,38 \euro        \\ \hline
TOTAL (sin IVA)                    & 31717,29 \euro        \\ \hline
IVA (22 \%)                        & 6977,80\euro        \\ \hline
TOTAL                              & 38695,10 \euro        \\ \hline
\end{tabular}
\caption{Importe del presupuesto total del proyecto}
\label{tab:presupuesto_total}
\end{table}

