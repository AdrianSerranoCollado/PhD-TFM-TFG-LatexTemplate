%%%%%%%%%%%%%%%%%%%%%%%%%%%%%%%%%%%%%%%%%%%%%%%%%%%%%%%%%%%%%%%%%%%%%%%%%%%
%
% Generic template for TFC/TFM/TFG/Tesis
%
% $Id: desarrollo.tex,v 1.4 2015/06/05 00:05:18 macias Exp $
%
% By:
%  + Javier Macías-Guarasa.
%    Departamento de Electrónica
%    Universidad de Alcalá
%  + Roberto Barra-Chicote.
%    Departamento de Ingeniería Electrónica
%    Universidad Politécnica de Madrid
% 
% Based on original sources by Roberto Barra, Manuel Ocaña, Jesús Nuevo, Pedro Revenga, Fernando Herránz and Noelia Hernández. Thanks a lot to all of them, and to the many anonymous contributors found (thanks to google) that provided help in setting all this up.
%
% See also the additionalContributors.txt file to check the name of additional contributors to this work.
%
% If you think you can add pieces of relevant/useful examples, improvements, please contact us at (macias@depeca.uah.es)
%
% You can freely use this template and please contribute with comments or suggestions!!!
%
%%%%%%%%%%%%%%%%%%%%%%%%%%%%%%%%%%%%%%%%%%%%%%%%%%%%%%%%%%%%%%%%%%%%%%%%%%%

\chapter{Desarrollo}
\label{cha:desarrollo}


\begin{FraseCelebre}
  \begin{Frase}
    A fuerza de construir bien, se llega a buen arquitecto\footnote{Tomado de ejemplos del proyecto \texis{}.}.
  \end{Frase}
  \begin{Fuente}
    Aristóteles
  \end{Fuente}
\end{FraseCelebre}


\section{Introducción}
\label{sec:introduccion-desarrollo}

En este capítulo se incluirá la descripción del desarrollo del trabajo.

El capítulo se estructura en n apartados....


\section{Desarrollo del sistema de experimentación}
\label{sec:desarr-del-sist}

Blah, blah, blah\ldots


\section{Planteamiento matemático}
\label{sec:libr-desarr}

También resulta útil poder introducir ecuaciones que se encuentran tanto en línea con el texto (como por ejemplo $\sigma=0.75$), como en un párrafo aparte (como en la ecuación \ref{eq:eq1}). Al igual que ocurre con las figuras, también se pueden referenciar las ecuaciones.

\begin{equation}
  \label{eq:eq1}
  p[q_t=\sigma_t|q_{t-1}=\sigma_{t-1}]
\end{equation}

\section{Conclusiones}
\label{sec:conclusiones-desarrollo}

Blah, blah, blah\ldots



%%% Local Variables:
%%% TeX-master: "../book"
%%% End:
