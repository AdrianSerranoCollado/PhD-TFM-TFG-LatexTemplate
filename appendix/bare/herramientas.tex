%%%%%%%%%%%%%%%%%%%%%%%%%%%%%%%%%%%%%%%%%%%%%%%%%%%%%%%%%%%%%%%%%%%%%%%%%%%
%
% Generic template for TFC/TFM/TFG/Tesis
%
% $Id: herramientas-bare.tex,v 1.1 2015/04/27 17:16:39 macias Exp $
%
% By:
%  + Javier Mac�as-Guarasa. 
%    Departamento de Electr�nica
%    Universidad de Alcal�
%  + Roberto Barra-Chicote. 
%    Departamento de Ingenier�a Electr�nica
%    Universidad Polit�cnica de Madrid   
% 
% Based on original sources by Roberto Barra, Manuel Oca�a, Jes�s Nuevo,
% Pedro Revenga, Fernando Herr�nz and Noelia Hern�ndez. Thanks a lot to
% all of them, and to the many anonymous contributors found (thanks to
% google) that provided help in setting all this up.
%
% See also the additionalContributors.txt file to check the name of
% additional contributors to this work.
%
% If you think you can add pieces of relevant/useful examples,
% improvements, please contact us at (macias@depeca.uah.es)
%
% You can freely use this template and please contribute with
% comments or suggestions!!!
%
%%%%%%%%%%%%%%%%%%%%%%%%%%%%%%%%%%%%%%%%%%%%%%%%%%%%%%%%%%%%%%%%%%%%%%%%%%%

\chapter{Herramientas y recursos}
\label{cha:herr-y-recurs}

Las herramientas necesarias para la elaboraci�n del proyecto han sido:

\begin{itemize}
\item PC compatible 
\item Sistema operativo GNU/Linux \cite{gnulinux}
\item Entorno de desarrollo Emacs \cite{emacs}
\item Entorno de desarrollo KDevelop \cite{kdevelop}
\item Procesador de textos \LaTeX \cite{lamport94}
\item Lenguaje de procesamiento matem�tico Octave  \cite{octave}
\item Control de versiones CVS \cite{cvs}
\item Compilador C/C++ gcc \cite{gcc}
\item Gestor de compilaciones make \cite{make}
\end{itemize}

%%% Local Variables:
%%% TeX-master: "../book"
%%% End:


