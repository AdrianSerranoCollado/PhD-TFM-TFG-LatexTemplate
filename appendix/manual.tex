%%%%%%%%%%%%%%%%%%%%%%%%%%%%%%%%%%%%%%%%%%%%%%%%%%%%%%%%%%%%%%%%%%%%%%%%%%%
%
% Generic template for TFC/TFM/TFG/Tesis
%
% $Id: manual.tex,v 1.4 2013/11/25 00:33:31 macias Exp $
%
% By:
%  + Javier Mac�as-Guarasa. 
%    Departamento de Electr�nica
%    Universidad de Alcal�
%  + Roberto Barra-Chicote. 
%    Departamento de Ingenier�a Electr�nica
%    Universidad Polit�cnica de Madrid   
% 
% Based on original sources by Roberto Barra, Manuel Oca�a, Jes�s Nuevo,
% Pedro Revenga, Fernando Herr�nz and Noelia Hern�ndez. Thanks a lot to
% all of them, and to the many anonymous contributors found (thanks to
% google) that provided help in setting all this up.
%
% If you think you can add pieces of relevant/useful examples,
% improvements, please contact us at (macias@depeca.uah.es)
%
% Copyleft 2013
%
%%%%%%%%%%%%%%%%%%%%%%%%%%%%%%%%%%%%%%%%%%%%%%%%%%%%%%%%%%%%%%%%%%%%%%%%%%%

\chapter{Manual de usuario}
\label{cha:concl-y-line}

\section{Introducci�n}
\label{sec:introapp1}


Introducci�n.

\section{Manual}
\label{sec:manual}

Pues eso.

\section{C�digo fuente}
\label{sec:codigo-fuente}

As� se inserta c�digo fuente, usando el estilo \texttt{CppExample} que
hemos definido en el preamble, escribiendo el c�digo directamente :

\begin{lstlisting}[style=CppExample]
#include <stdio.h>

// Esto es una funci�n de prueba
void funcionPrueba(int argumento)
{	
	int prueba = 1;

  printf("Esto es una prueba [%d][%d]\n", argumento, prueba);

}
\end{lstlisting}

O bien insertando directamente c�digo de un fichero externo, como en el
ejemplo \ref{sample1}, usando
\texttt{\textbackslash{}lstinputlisting} y cambiando el estilo a
\texttt{cCbluebox} (adem�s de usar el entorno \texttt{codefloat} para
evitar pagebreaks, definir captions, etc.):

\begin{codefloat}
\label{sample1}
\caption{Ejemplo de c�digo fuente con un \texttt{lstinputlisting} dentro de un \texttt{codefloat}.}
\lstinputlisting[style=Cbluebox]{appendix/function.c}
\end{codefloat}


O por ejemplo en matlab, definiendo settings en lugar de usar estilos
definidos:

\lstset{language=matlab}
\lstset{tabsize=2}
\lstset{commentstyle=\textit}
\lstset{stringstyle=\ttfamily, basicstyle=\small}
\begin{lstlisting}[frame=trbl]{}
%
% add_simple.m - Simple matlab script to run with condor
%
a = 9;
b = 10;

c = a+b;

fprintf(1, 'La suma de %d y %d es igual a %d\n', a, b, c);
\end{lstlisting}

O incluso como en el listado \ref{sample2}, usando un layout m�s refinado (con
los settings de \url{http://www.rafalinux.com/?p=599} en un lststyle
\texttt{Cnice}):


\begin{codefloat}
\label{sample2}
\caption{Ejemplo de c�digo fuente con estilo \texttt{Cnice}, de nuevo
  con un \texttt{lstinputlisting} dentro de un \texttt{codefloat}.}
\lstinputlisting[style=Cnice]{appendix/hello.c}
\end{codefloat}

\noindent
Ahora compila usando \texttt{gcc}:


\begin{lstlisting}[style=console, numbers=none]
$ gcc  -o hello hello.c
\end{lstlisting}


%%% Local Variables:
%%% TeX-master: "../book"
%%% End:


