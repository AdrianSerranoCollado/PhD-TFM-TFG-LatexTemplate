 % This file is public domain
 % If you want to use arara, you need the following directives:
 % arara: pdflatex: { synctex: on }
 % arara: makeglossaries
 % arara: pdflatex: { synctex: on }
 % arara: pdflatex: { synctex: on }
\documentclass[a4paper]{report}

\usepackage[colorlinks,plainpages=false]{hyperref}
\usepackage[acronym,% create 'acronym' glossary type
            nomain,% 'main' glossary not needed as using 'acronym'
            style=altlist, % use altlist style
            toc, % add the glossary to the table of contents
           ]{glossaries}

\makeglossaries

\setacronymstyle{long-sc-short-desc}

\renewcommand*{\glsseeitemformat}[1]{\acronymfont{\glsentrytext{#1}}}


\renewcommand*{\glsnamefont}[1]{\textmd{#1}}

\newacronym[description={Statistical pattern recognition
technique~\protect\cite{svm}}, % acronym's description
]{svm}{svm}{support vector machine}

\newacronym
 [description={Statistical pattern recognition technique
  using the ``kernel trick''},% acronym's description
  see={[see also]{svm}},
]
{ksvm}{ksvm}{kernel support vector machine}

\begin{document}
\tableofcontents

\chapter{Support Vector Machines}

The \gls{svm} is used widely in the area of pattern recognition.
 % plural form with initial letter in uppercase:
\Glspl{svm} are \ldots\ (but beware, converting the initial letter to
upper case for a small caps acronym is sometimes considered
poor style).

Short version: \acrshort{svm}. Long version: \acrlong{svm}. Full
version: \acrfull{svm}. Description: \glsentrydesc{svm}.

This is the entry in uppercase: \GLS{svm}.

\chapter{Kernel Support Vector Machines}

The \gls{ksvm} is \ifglsused{svm}{an}{a} \gls{svm} that uses
the so called ``kernel trick''. This is the entry's description without
a link: \glsentrydesc{ksvm}.

\glsresetall
Possessive: \gls{ksvm}['s].
Make the glossary entry number bold for this
one \gls[format=hyperbf]{svm}.

\begin{thebibliography}{1}
\bibitem{svm} \ldots
\end{thebibliography}

\printglossary

\end{document}
