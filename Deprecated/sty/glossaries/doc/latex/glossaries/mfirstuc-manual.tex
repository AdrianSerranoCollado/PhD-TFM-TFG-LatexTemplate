\documentclass{nlctdoc}

\usepackage{alltt}
\usepackage{mfirstuc}
\usepackage{pifont}
\usepackage[utf8]{inputenc}
\ifpdf
 \usepackage[T1]{fontenc}
 \usepackage{metalogo}
\else
 \providecommand{\XeLaTeX}{XeLaTeX}
 \providecommand{\XeTeX}{XeTeX}
\fi
\usepackage{cmap}
\usepackage[colorlinks,
            bookmarks,
            hyperindex=false,
            pdfauthor={Nicola L.C. Talbot},
            pdftitle={mfirstuc.sty: uppercasing first letter},
            pdfkeywords={LaTeX,package,uppercase}]{hyperref}

\begin{document}
 \title{mfirstuc.sty v1.09: 
uppercasing first letter}
 \author{Nicola L.C. Talbot\\[10pt]
Dickimaw Books\\
\url{http://www.dickimaw-books.com/}}

 \date{2014-07-30}
 \maketitle
 \tableofcontents

 \section{Introduction}
 \label{sec:intro}

The \styfmt{glossaries} bundle includes the package 
\sty{mfirstuc} which provides the command:
\begin{definition}[\DescribeMacro{\makefirstuc}]
\cs{makefirstuc}\marg{stuff}
\end{definition}
This makes the first object of
\meta{stuff} uppercase unless \meta{stuff} starts with a control
sequence followed by a non-empty group, in which case the first
object in the group is converted to uppercase.
\textbf{No expansion is performed on the argument.}

Examples:
\begin{itemize}
\item |\makefirstuc{abc}| produces \makefirstuc{abc}.

\item |\makefirstuc{\emph{abc}}| produces \makefirstuc{\emph{abc}}
(\ics{MakeUppercase} has been applied to the letter \qt{a} rather
than \cs{emph}). Note however that
\begin{verbatim}
\makefirstuc{{\em abc}}
\end{verbatim}
produces \makefirstuc{{\em abc}} (first object is |{\em abc}| so
equivalent to |\MakeUppercase{\em abc}|), and
\begin{verbatim}
{\makefirstuc{\em abc}}
\end{verbatim}
produces {\makefirstuc{\em abc}} (|\em| doesn't have an argument
therefore first object is |\em| and so is equivalent to
|{\MakeUppercase{\em}abc}|).

\item |\makefirstuc{{\'a}bc}| produces \makefirstuc{{\'a}bc}.

\item |\makefirstuc{\ae bc}| produces \makefirstuc{\ae bc}.

\item |\makefirstuc{{\ae}bc}| produces \makefirstuc{{\ae}bc}.

\item |\makefirstuc{{À}bc}| produces \makefirstuc{{À}bc}.

\end{itemize}
Note that non-Latin or accented characters appearing at the
start of the text must be placed in a group (even if you are
using the \sty{inputenc} package). The reason for this restriction
is detailed in \sectionref{sec:utf8}.

\begin{important}
In version 1.02 of \styfmt{mfirstuc}, a bug fix resulted in a change
in output if the first object is a control sequence followed by
an empty group. Prior to version 1.02, |\makefirstuc{\ae{}bc}|
produced \ae Bc. However as from version 1.02, it now produces
\AE bc.
\end{important}

Note also that
\begin{verbatim}
\newcommand{\abc}{abc}
\makefirstuc{\abc}
\end{verbatim}
produces: ABC. This is because the first object in the argument of
\cs{makefirstuc} is \cs{abc}, so it does |\MakeUppercase{\abc}|.
Whereas:
\begin{verbatim}
\newcommand{\abc}{abc}
\expandafter\makefirstuc\expandafter{\abc}
\end{verbatim}
produces: Abc. There is a short cut command which will do this:
\begin{definition}[\DescribeMacro{\xmakefirstuc}]
\cs{xmakefirstuc}\marg{stuff}
\end{definition}
This is equivalent to \cs{expandafter}\cs{makefirstuc}\cs{expandafter}\marg{stuff}. So
\begin{verbatim}
\newcommand{\abc}{abc}
\xmakefirstuc{\abc}
\end{verbatim}
produces: 
\newcommand{\abc}{abc}%
\xmakefirstuc{\abc}.

\begin{important}
\cs{xmakefirstuc} only performs one level expansion on the
\emph{first} object in its argument. It does not fully expand the entire
argument.
\end{important}

If you use \styfmt{mfirstuc} without the \sty{glossaries} package, 
the standard \cs{MakeUppercase} command is used. If used with
\sty{glossaries}, \ics{MakeTextUppercase} (defined by \sty{textcase}
the package) is used instead. If you are using \styfmt{mfirstuc}
without the \styfmt{glossaries} package and want to use
\ics{MakeTextUppercase} instead, you can redefine
\begin{definition}[\DescribeMacro\glsmakefirstuc]
\cs{glsmakefirstuc}\marg{text}
\end{definition}
For example:
\begin{verbatim}
\renewcommand{\glsmakefirstuc}[1]{\MakeTextUppercase #1}
\end{verbatim}
Remember to also load \sty{textcase} (\styfmt{glossaries} loads this
automatically).

\section{Capitalise Each Word in a Phrase or Sentence (Title Case)}
\label{sec:capitalisewords}

New to mfirstuc v1.06:
\begin{definition}[\DescribeMacro{\capitalisewords}]
\cs{capitalisewords}\marg{text}
\end{definition}
This command apply \ics{makefirstuc} to each word in \meta{text}
where the space character is used as the word separator. Note that
it has to be a plain space character, not another form of space,
such as \verb|~| or \cs{space}. Note that no expansion is performed
on \meta{text}.

\begin{definition}[\DescribeMacro{\xcapitalisewords}]
\cs{xcapitalisewords}\marg{text}
\end{definition}
This is a short cut for
\cs{expandafter}\cs{capitalisewords}\cs{expandafter}\marg{text}.

\begin{important}
If you are using \sty{hyperref} and want to use \cs{capitalisewords}
or \ics{makefirstuc} (or \cs{xcapitalisewords}\slash\cs{xmakefirstuc}) 
in a section heading, the PDF bookmarks won't be able to use the command 
as it's not expandable, so you will get a warning that looks like:
\begin{verbatim}
Package hyperref Warning: Token not allowed in a PDF string 
(PDFDocEncoding):
(hyperref)                removing `\capitalisewords'
\end{verbatim}
If you want to provide an alternative for the PDF bookmark, you can
use \sty{hyperref}'s \ics{texorpdfstring} command. See the
\sty{hyperref} manual for further details.
\end{important}

Examples:
\begin{enumerate}
\item 
\begin{verbatim}
\capitalisewords{a book of rhyme.}
\end{verbatim}
produces:
\capitalisewords{a book of rhyme.}

\item
\begin{verbatim}
\capitalisewords{a book\space of rhyme.}
\end{verbatim}
produces:
\capitalisewords{a book\space of rhyme.}

\item
\begin{verbatim}
\newcommand{\mytitle}{a book\space of rhyme.}
\capitalisewords{\mytitle}
\end{verbatim}
produces:
\newcommand{\mytitle}{a book\space of rhyme.}
\capitalisewords{\mytitle}
(No expansion is performed on \cs{mytitle}, so \meta{text}
consists of just one ``word''.) Compare with next example:

\item
\begin{verbatim}
\newcommand{\mytitle}{a book\space of rhyme.}
\xcapitalisewords{\mytitle}
\end{verbatim}
produces:
\xcapitalisewords{\mytitle}

\end{enumerate}

As from v1.09, you can specify words which shouldn't be capitalised unless they
occur at the start of \meta{text} using:
\begin{definition}[\DescribeMacro]
\cs{MFUnocap}\marg{word}
\end{definition}
This only has a local effect. The global version is:
\begin{definition}[\DescribeMacro]
\cs{gMFUnocap}\marg{word}
\end{definition}

For example:
\begin{verbatim}
\capitalisewords{the wind in the willows}

\MFUnocap{in}%
\MFUnocap{the}%

\capitalisewords{the wind in the willows}
\end{verbatim}
produces:
\begin{display}
\capitalisewords{the wind in the willows}

\MFUnocap{in}%
\MFUnocap{the}%

\capitalisewords{the wind in the willows}
\end{display}
The list of words that shouldn't be capitalised can be cleared using
\begin{definition}[\DescribeMacro\MFUclear]
\cs{MFUclear}
\end{definition}

The package \sty{mfirstuc-english} loads \sty{mfirstuc} and uses
\cs{MFUnocap} to add common English articles and conjunctions, such
as ``a'', ``an'', ``and'', ``but''. You may want to add other
words to this list, such as prepositions, but as there's some
dispute over whether prepositions should be capitalised, I~don't
intend to add them to this package.

If you want to write a similar package for another language, all you
need to do is create a file with the extension \texttt{.sty}
that starts with
\begin{verbatim}
\NeedsTeXFormat{LaTeX2e}
\end{verbatim}
The next line should identify the package. For example, if you have
called the file \texttt{mfirstuc-french.sty} then you need:
\begin{verbatim}
\ProvidesPackage{mfirstuc-french}
\end{verbatim}
It's a good idea to also add a version in the final optional
argument, for example:
\begin{verbatim}
\ProvidesPackage{mfirstuc-french}[2014/07/30 v1.0]
\end{verbatim}
Next load \sty{mfirstuc}:
\begin{verbatim}
\RequirePackage{mfirstuc}
\end{verbatim}
Now add all your \cs{MFUnocap} commands. For example:
\begin{verbatim}
\MFUnocap{de}
\end{verbatim}
At the end of the file add:
\begin{verbatim}
\endinput
\end{verbatim}

Put the file somewhere on \TeX's path, and now you can use this
package in your document. You might also consider uploading it
to CTAN in case other users find it useful.

\section{UTF-8}
\label{sec:utf8}

The \cs{makefirstuc} command works by utilizing the fact that, in
most cases, \TeX\ doesn't require a regular argument to be enclosed
in braces if it only consists of a single token. (This is why you
can do, say, \verb|\frac12| instead of \verb|\frac{1}{2}| or
\verb|x^2| instead of \verb|x^{2}|, although some users
frown on this practice.)

A~simplistic version of the \cs{makefirstuc} command is:
\begin{verbatim}
\newcommand*{\FirstUC}[1]{\MakeUppercase #1}
\end{verbatim}
Here \verb|\FirstUC{abc}| is equivalent to \verb|\MakeUppercase abc|
and since \cs{MakeUppercase} requires an argument, it grabs the
first token (the character ``a'' in this case) and uses that as the
argument so that the result is: Abc.

The \sty{glossaries} package needs to take into account the fact
that the text may be contained in the argument of a formatting
command, such as \cs{acronymfont}, so \cs{makefirstuc} has to be
more complicated than the trivial \cs{FirstUC} shown above, but at
its basic level, \cs{makefirstuc} uses this same method and is the
reason why, in most cases, you don't need to enclose the first
character in braces. So if \verb|\MakeUppercase |\meta{stuff} works,
then \verb|\makefirstuc|\marg{stuff} should also work and so should
\verb|\makefirstuc{\foo|\marg{stuff}\verb|}|, but if 
\verb|\MakeUppercase |\meta{stuff} doesn't work, then neither will
\verb|\makefirstuc|\marg{stuff} 
(or \verb|\makefirstuc{\foo|\marg{stuff}\verb|}|).

Try the following document:
\begin{alltt}
\cs{documentclass}\{article\}

\cs{usepackage}[utf8]\{inputenc\}
\cs{usepackage}[T1]\{fontenc\}

\cs{begin}\{document\}

\cs{MakeUppercase} \~abc

\cs{end}\{document\}
\end{alltt}

This will result in the error:
\begin{verbatim}
! Argument of \UTFviii@two@octets has an extra }.
\end{verbatim}
This is why \verb|\makefirstuc{|\texttt{\~abc}\verb|}| won't work.
It will only work if the character \texttt{\~a} is placed inside a
group.

The reason for this error message is due to \TeX\ having been written before
Unicode was invented. Although \texttt{\~a} may look like a single
character in your text editor, from \TeX's point of view it's \emph{two} 
tokens. So
\begin{alltt}
\cs{MakeUppercase} \~abc
\end{alltt}
tries to apply \cs{MakeUppercase} to just the first octet of \~a.
This means that the second octet has been separated from the first octet,
which is the cause of the error. In this case the argument isn't a
single token, so the two tokens (the first and second octet of \~a)
must be grouped:
\begin{alltt}
\cs{MakeUppercase}\{\~a\}bc
\end{alltt}

Note that \XeTeX\ (and therefore \XeLaTeX) is a modern
implementation of \TeX\ designed to work with Unicode and therefore
doesn't suffer from this drawback. Now let's look at the \XeLaTeX\
equivalent of the above example:
\begin{alltt}
\cs{documentclass}\{article\}

\cs{usepackage}\{fontspec\}

\cs{begin}\{document\}

\cs{MakeUppercase} \~abc

\cs{end}\{document\}
\end{alltt}

This works correctly when compiled with \XeLaTeX. This means
that \cs{makefirstuc}\verb|{|\texttt{\~abc}\verb|}| will work
\emph{provided you use \XeLaTeX\ and the \sty{fontspec} package}.

\end{document}
