 % This file is public domain
 % If you want to use arara, you need the following directives:
 % arara: pdflatex: { synctex: on }
 % arara: makeglossaries
 % arara: pdflatex: { synctex: on }
 % arara: pdflatex: { synctex: on }
\documentclass[a4paper]{report}

\usepackage[plainpages=false,colorlinks]{hyperref}
\usepackage[toc,order=word,subentrycounter]{glossaries}

\usepackage{glossary-inline}

\renewcommand*{\glossarysection}[2][]{\textbf{#1}: }
\renewcommand{\glsnamefont}[1]{\textsc{#1}}
\setglossarystyle{inline}

\makeglossaries

\newglossaryentry{Perl}{name=\texttt{Perl},
sort=Perl, % need a sort key because name contains a command
description=A scripting language}

\newglossaryentry{glossary}{name=glossary,
description={\nopostdesc},
plural={glossaries}}

\newglossaryentry{glossarycol}{
description={collection of glosses},
sort={2},
parent={glossary}}

\newglossaryentry{glossarylist}{
description={list of technical words},
sort={1},
parent={glossary}}

\newglossaryentry{pagelist}{name=page list,
 % description value has to be enclosed in braces
 % because it contains commas
description={a list of individual pages or page ranges
(e.g.\ 1,2,4,7-9)}}

\newglossaryentry{mtrx}{name=matrix,
description={rectangular array of quantities},
 % plural is not simply obtained by appending an s, so specify
plural=matrices}

 % entry with a paragraph break in the description

\newglossaryentry{par}{name=paragraph,
description={distinct section of piece of
writing. Beginning on new, usually indented, line}}

 % entry with two types of plural. Set the plural form to the
 % form most likely to be used. If you want to use a different
 % plural, you will need to explicity specify it in \glslink
\newglossaryentry{cow}{name=cow,
 % this isn't necessary, as this form (appending an s) is
 % the default
plural=cows,
 % description:
description={(\emph{pl.}\ cows, \emph{archaic} kine) an adult
female of any bovine animal}}

\newglossaryentry{bravo}{name={bravo},
description={\nopostdesc}}

\newglossaryentry{bravo1}{description={cry of approval (pl.\ bravos)},
sort={1},
plural={bravos},
parent=bravo}

\newglossaryentry{bravo2}{description={hired ruffian or killer (pl.\ bravoes)},
sort={2},
plural={bravoes},
parent=bravo}

\newglossaryentry{seal}{name=seal,description={sea mammal with
flippers that eats fish}}

\newglossaryentry{sealion}{name={sea lion},
description={large seal}}

\begin{document}

\title{Sample Document Using glossary Package}
\author{Nicola Talbot}
\pagenumbering{alph}% prevent duplicate page link names if using PDF
\maketitle

\pagenumbering{roman}
\tableofcontents

\chapter{Introduction}
\pagenumbering{arabic}

This document has a glossary in a footnote\footnote{\printglossaries}.

A \gls{glossarylist} (definition~\glsrefentry{glossarylist}) is a
very useful addition to any technical document, although a
\gls{glossarycol} (definition~\glsrefentry{glossarycol}) can also
simply be a collection of glosses, which is another thing entirely.
Some documents have multiple \glspl{glossarylist}.

Once you have run your document through \LaTeX, you
will then need to run the \texttt{.glo} file through
\texttt{makeindex}.  You will need to set the output
file so that it creates a \texttt{.gls} file instead
of an \texttt{.ind} file, and change the name of
the log file so that it doesn't overwrite the index
log file (if you have an index for your document).
Rather than having to remember all the command line
switches, you can call the \gls{Perl} script
\texttt{makeglossaries} which provides a convenient
wrapper.

If a comma appears within the name or description, grouping
must be used, e.g.\ in the description of \gls{pagelist}.

\chapter{Plurals and Paragraphs}

Plurals are assumed to have the letter s appended, but if this is
not the case, as in \glspl{mtrx}, then you need to specify the
plural when you define the entry. If a term may have multiple
plurals (for example \glspl{cow}/\glslink{cow}{kine}) then
define the entry with the plural form most likely to be used and
explicitly specify the alternative form using \verb|\glslink|.
\Glspl{seal} and \glspl{sealion} have regular plural forms.

\Gls{bravo} is a homograph, but the plural forms are spelt
differently. The plural of \gls{bravo1}, a cry of approval
(definition~\glsrefentry{bravo1}), is \glspl{bravo1}, whereas the
plural of \gls{bravo2}, a hired ruffian or killer
(definition~\glsrefentry{bravo2}), is \glspl{bravo2}.

\Glspl{par} can cause a problem in commands, so care is needed
when having a paragraph break in a \gls{glossarylist} entry.

\chapter{Ordering}

There are two types of ordering: word ordering (which places
``\gls{sealion}'' before ``\gls{seal}'') and letter ordering
(which places ``\gls{seal}'' before ``\gls{sealion}'').

\end{document}
