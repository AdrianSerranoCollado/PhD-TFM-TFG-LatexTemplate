%%%%%%%%%%%%%%%%%%%%%%%%%%%%%%%%%%%%%%%%%%%%%%%%%%%%%%%%%%%%%%%%%%%%%%%%%%%
%
% Generic template for TFC/TFM/TFG/Tesis
%
% $Id: preamble-letter.tex,v 1.7 2020/03/24 17:18:25 macias Exp $
%
% By:
%  + Javier Macías-Guarasa. 
%    Departamento de Electrónica
%    Universidad de Alcalá
%  + Roberto Barra-Chicote. 
%    Departamento de Ingeniería Electrónica
%    Universidad Politécnica de Madrid   
% 
% Based on original sources by Roberto Barra, Manuel Ocaña, Jesús Nuevo,
% Pedro Revenga, Fernando Herránz and Noelia Hernández. Thanks a lot to
% all of them, and to the many anonymous contributors found (thanks to
% google) that provided help in setting all this up.
%
% See also the additionalContributors.txt file to check the name of
% additional contributors to this work.
%
% If you think you can add pieces of relevant/useful examples,
% improvements, please contact us at (macias@depeca.uah.es)
%
% You can freely use this template and please contribute with
% comments or suggestions!!!
%
%%%%%%%%%%%%%%%%%%%%%%%%%%%%%%%%%%%%%%%%%%%%%%%%%%%%%%%%%%%%%%%%%%%%%%%%%%%


\synctex=1

%\usepackage[a4,cam,center]{crop}
%\crop[font=\upshape\mdseries\small\textsf]

% ifthen to allow using language dependent settings
\usepackage{ifthen}
\usepackage{iftex}
\ifPDFTeX
  \usepackage[utf8]{inputenc} % Para poder escribir con acentos y ñ.
  \usepackage[T1]{fontenc}      % Para que haga bien la ``hyphenation''. No
\fi                                % usar si no es necesario, porque ralentiza muchisimo la compilación.
\usepackage{ae}               % Para que todas las fuentes sean Type1, y ninguna Type3.

\usepackage[spanish, english]{babel}

% Use this if you want to delete headers and footers in empty pages
\usepackage{emptypage}


\usepackage{wasysym}

\usepackage{tabularx}

\usepackage{geometry}
\geometry{verbose,a4paper,tmargin=2.5cm,bmargin=2.5cm,lmargin=2.5cm,rmargin=2.5cm}

\usepackage{enumitem}

\usepackage{graphicx}
\usepackage{float}

\usepackage{hyperxmp}
\usepackage[
%% ps2pdf,                %%% hyper-references for ps2pdf
bookmarks=true,%                   %%% generate bookmarks ...
bookmarksnumbered=true,            %%% ... with numbers
hypertexnames=false,               %%% needed for correct links to
%%% figures!!!
% hypertexnames=true,               %%% needed for correct links on pagebackrefs!!!
breaklinks=true,                   %%% breaks lines, but links are very small
% pagebackref=true,
% linktocpage=true,                 %%% enlace en el numero de página.
linktoc=all,
colorlinks=true,
linkcolor=blue,    
citecolor=green,
urlcolor=blue,                     %%% texto  con color (further
%%% modified in myconfig.tex)
% linkbordercolor={0 0 1},           %%% blue frames around links
pdfborder={0 0 112.0},              %%% border-width of frames 
hyperfootnotes=false,
]{hyperref}                        %%% will be multiplied with 0.009 by ps2pdf

% To uppercase first letter
\usepackage{mfirstuc}

%%%%%%%%%%%%%%%%%%%%%%%%%%%%%%%%%%%%%%%%%%%%%%%%%%%%%%%%%%%%%%%%%%%%%%%%%%%
% To allow checking for initial letter of string being a given one
\usepackage{xstring}

%%%%%%%%%%%%%%%%%%%%%%%%%%%%%%%%%%%%%%%%%%%%%%%%%%%%%%%%%%%%%%%%%%%%%%%%%%%
% To allow bold + tt (from https://tex.stackexchange.com/questions/215482/how-do-i-get-texttt-with-bold-face-in-latex)
\usepackage{bold-extra}

%%% Local Variables:
%%% TeX-master: "../book"
%%% End:



