 % This file is public domain
 %
 % See also sampleAcrDesc.tex
 %
 % If you want to use arara, you need the following directives:
 % arara: pdflatex: { synctex: on }
 % arara: makeglossaries
 % arara: pdflatex: { synctex: on }
 % arara: pdflatex: { synctex: on }
\documentclass[a4paper]{report}

\usepackage[colorlinks,plainpages=false]{hyperref}

\usepackage[style=altlist, % use altlist style
            toc % add the glossary to the table of contents
           ]{glossaries}

\makeglossaries

\newglossaryentry{svm}{
 % how the entry name should appear in the glossary
name={Support vector machine (SVM)},
 % how the description should appear in the glossary
description={Statistical pattern recognition
technique~\cite{svm}},
 % how the entry should appear in the document text
text={svm},
 % how the entry should appear the first time it is
 % used in the document text
first={support vector machine (svm)}}

\newglossaryentry{ksvm}{
name={Kernel support vector machine (KSVM)},
description={Statistical pattern recognition technique
using the ``kernel trick'' (see also SVM)},
text={ksvm},
first={kernel support vector machine}}

\begin{document}
\tableofcontents

\chapter{Support Vector Machines}

The \gls{svm} is used widely in the area of pattern recognition.
 % plural form with initial letter in uppercase:
\Glspl{svm} are \ldots

This is the text produced without a link: \glsentrytext{svm}.
This is the text produced on first use without a link:
\glsentryfirst{svm}. This is the entry's description without
a link: \glsentrydesc{svm}.

This is the entry in uppercase: \GLS{svm}.

\chapter{Kernel Support Vector Machines}

The \gls{ksvm} is \ifglsused{svm}{an}{a} \gls{svm} that uses
the so called ``kernel trick''.

\glsresetall
Possessive: \gls{ksvm}['s].
Make the glossary entry number bold for this
one \gls[format=hyperbf]{svm}.

\begin{thebibliography}{1}
\bibitem{svm} \ldots
\end{thebibliography}

\printglossary[title={Acronyms}]

\end{document}
