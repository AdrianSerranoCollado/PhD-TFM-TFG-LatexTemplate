\documentclass{nlctdoc}

\usepackage[utf8]{inputenc}
\ifpdf
\usepackage[T1]{fontenc}
\usepackage{mathpazo}
\usepackage[scaled=.88]{helvet}
\usepackage{courier}
\usepackage{xr-hyper}
\fi

\usepackage{alltt}
\usepackage{pifont}
\usepackage{array}

\usepackage[colorlinks,pdfauthor={Nicola L.C. Talbot},
            pdftitle={The glossaries package: a guide for beginners}]{hyperref}
\usepackage[nomain]{glossaries}

\newcommand*{\yes}{\ding{52}}
\newcommand*{\no}{\ding{56}}

\title{The glossaries package v4.11: 
a guide for beginners}
\author{Nicola L.C. Talbot}
\date{2014-09-01}

\ifpdf
  \externaldocument{glossaries-user}
\fi

\makeatletter
\newcommand*{\optionlabel}[1]{%
 \@glstarget{option#1}{}\textbf{Option~#1}}
\makeatother

\newcommand*{\opt}[1]{\hyperlink{option#1}{Option~#1}}
\newcommand*{\optsor}[2]{Options~\hyperlink{option#1}{#1}
or~\hyperlink{option#2}{#2}}
\newcommand*{\optsand}[2]{Options~\hyperlink{option#1}{#1}
and~\hyperlink{option#2}{#2}}

\begin{document}
\maketitle

\begin{abstract}
The \styfmt{glossaries} package is very flexible, but this means
that it has a lot options, and since a user guide is supposed to
provide a complete list of all the high-level user commands, the main
user manual is quite big. This can be rather
daunting for beginners, so this document is a brief introduction
just to help get you started. If you find yourself saying, ``Yeah,
but how can I do\ldots?'' then it's time to move on to the 
\docref{main user manual}{glossaries-user}.

I've made some statements in this document that don't actually tell
you the full truth, but it would clutter the document and cause
confusion if I keep writing ``except when \ldots'' or ``but you can
also do this, that or the other'' or ``you can do it this way but
you can also do it that way, but that way may cause complications
under certain circumstances''.
\end{abstract}

\tableofcontents

\section{Getting Started}
\label{sec:start}

As with all packages, you need to load \styfmt{glossaries} with
\cs{usepackage}, but there are certain packages that must be loaded
before \styfmt{glossaries}, if they are required: \sty{hyperref},
\sty{babel}, \sty{polyglossia}, \sty{inputenc} and \sty{fontenc}.
(You don't have to load these packages, but if you want them, you
must load them before \styfmt{glossaries}.)

Once you have loaded \styfmt{glossaries}, you need to define
your terms in the preamble and then you can use them throughout the
document. Here's a simple example:
\begin{verbatim}
\documentclass{article}

\usepackage{glossaries}

\newglossaryentry{ex}{name={sample},description={an example}}

\begin{document}
Here's my \gls{ex} term.
\end{document}
\end{verbatim}
This produces:
\begin{display}
Here's my sample term.
\end{display}
Here's another example:
\begin{verbatim}
\documentclass{article}

\usepackage{glossaries}

\setacronymstyle{long-short}

\newacronym{svm}{SVM}{support vector machine}

\begin{document}
First use: \gls{svm}. Second use: \gls{svm}.
\end{document}
\end{verbatim}
This produces:
\begin{display}
First use: support vector machine (SVM). Second use: SVM.
\end{display}
In this case, the text produced by \verb|\gls{svm}| changed after the first
use. The first use produced the long form followed by the short form
in parentheses because I set the acronym style to
\texttt{long-short}. I suggest you try the above two examples to
make sure you have the package correctly installed.
If you get an \texttt{undefined control sequence} error, check that
the version number at the top of this document matches the version
you have installed. (Open the \texttt{.log} file and search for the
line that starts with \texttt{Package: glossaries} followed by a
date and version.) 

If you like, you can put all your definitions in another file (for
example, \texttt{defns.tex}) and load that file in the preamble using
\cs{loadglsentries} with the filename as the argument. For example:
\begin{verbatim}
\loadglsentries{defns}
\end{verbatim}

Don't try inserting formatting commands into the definitions as they
can interfere with the underlying mechanism. Instead, the
formatting should be done by the style. For example, suppose I~want
to replace \texttt{SVM} in the above to \verb|\textsc{svm}|, then
I~need to select a style that uses \cs{textsc}, like this:
\begin{verbatim}
\documentclass{article}

\usepackage{glossaries}

\setacronymstyle{long-sc-short}

\newacronym{svm}{svm}{support vector machine}

\begin{document}
First use: \gls{svm}. Second use: \gls{svm}.
\end{document}
\end{verbatim}

As you can hopefully see from the above examples, there are two main ways of
defining a term. In both cases, the term is given
a label. In the first case the label was \texttt{ex} and in the
second case the label was \texttt{svm}. The label is just a way of
identifying the term (like the standard \cs{label}\slash\cs{ref}
mechanism). It's best to just use the following alphanumerics in the
labels: \texttt{a}, \ldots, \texttt{z}, \texttt{A}, \ldots,
\texttt{Z}, \texttt{0}, \ldots, \texttt{9}.  Sometimes you can also
use punctuation characters but not if another package (such as
\sty{babel}) meddles with them. Don't try using any characters
outside of the basic Latin set (for example, \'e or \ss) or things
will go horribly wrong.  This warning only applies to the label. It
doesn't apply to the text that appears in the document.

\begin{important}
Don't use \cs{gls} in chapter or section headings as it can have
some unpleasant side-effects. Instead use \cs{glsentrytext} for
regular entries and one of \cs{glsentryshort}, \cs{glsentrylong}
or \cs{glsentryfull} for acronyms.
\end{important}

The above examples are reasonably straightforward. The difficulty
comes if you want to display a sorted list of all the entries you
have used in the document. The \styfmt{glossaries} package provides
three options: use \TeX\ to perform the sorting; use
\texttt{makeindex} to perform the sorting; use \texttt{xindy} to
perform the sorting.

The first option (using \TeX) is the simplest and easiest method,
but it's inefficient. To use this method, add \cs{makenoidxglossaries}
to the preamble and put \cs{printnoidxglossaries} at the place where
you want your glossary. For example:
\begin{verbatim}
\documentclass{article}

\usepackage{glossaries}

\makenoidxglossaries

\newglossaryentry{potato}{name={potato},plural={potatoes},
 description={starchy tuber}}

\newglossaryentry{cabbage}{name={cabbage},
 description={vegetable with thick green or purple leaves}}

\newglossaryentry{carrot}{name={carrot},
 description={orange root}}

\begin{document}
Chop the \gls{cabbage}, \glspl{potato} and \glspl{carrot}.

\printnoidxglossaries
\end{document}
\end{verbatim}
Try this out and run \LaTeX\ (or pdf\LaTeX) \emph{twice}. The first
run won't show the glossary. It will only appear on the second run.
The glossary has a vertical gap between the ``carrot'' term and the
``potato'' term. This is because the entries in the glossaries are
grouped according to their first letter. If you don't want this gap,
just add \texttt{nogroupskip} to the package options:
\begin{verbatim}
\usepackage[nogroupskip]{glossaries}
\end{verbatim}
If you try out this example you may also notice that the description
is followed by a full stop (period) and a number. The number is the
location in the document where the entry was used, so you can lookup
the term in the glossary and be directed to the relevant pages. It
may be that you don't want this back-reference, in which case you
can suppress it using the \texttt{nonumberlist} package option:
\begin{verbatim}
\usepackage[nonumberlist]{glossaries}
\end{verbatim}
If you don't like the terminating full stop, you can suppress that
with the \texttt{nopostdot} package option:
\begin{verbatim}
\usepackage[nopostdot]{glossaries}
\end{verbatim}

You may have noticed that I've used another command in the above example:
\cs{glspl}. This displays the plural form. By default, this is just
the singular form with the letter ``s'' appended, but in the case of
``potato'' I had to specify the correct plural using the
\texttt{plural} key.

As I mentioned earlier, using \TeX\ to sort the entries is the
simplest but least efficient method. If you have a large glossary or
if your terms contain non-Latin characters, then you will have a much
faster build time if you use \texttt{makeindex} or \texttt{xindy}.
If you are using extended Latin or non-Latin characters, then
\texttt{xindy} is the recommended method. These two methods are
described in more detail in \sectionref{sec:printglossaries}.

The rest of this document briefly describes the main commands
provided by the \styfmt{glossaries} package.

\section{Defining Terms}
\label{sec:defterm}

When you use the \sty{glossaries} package, you need to define
glossary entries in the document preamble. These entries could be
a~word, phrase, acronym or symbol. They're usually accompanied by
a~description, which could be a short sentence or an in-depth
explanation that spans multiple paragraphs. The simplest method of
defining an entry is to use:
\begin{definition}
\begin{alltt}
\cs{newglossaryentry}\marg{label}
\verb|{|
  name=\marg{name},
  description=\marg{description},
  \meta{other options}
\verb|}|
\end{alltt}
\end{definition}
where \meta{label} is a unique label that identifies this entry.
(Don't include the angle brackets \meta{ }. They just indicate the parts of
the code you need to change when you use this command in your document.) 
The \meta{name} is the word, phrase or symbol you are defining,
and \meta{description} is the description to be displayed in the
glossary.

This command is a ``short'' command, which means that
\meta{description} can't contain a~paragraph break. If you have
a~long description, you can instead use:
\begin{definition}
\begin{alltt}
\cs{longnewglossaryentry}\marg{label}
\verb|{|
  name=\marg{name},
  \meta{other options}
\verb|}|
\marg{description}
\end{alltt}
\end{definition}

Examples:
\begin{enumerate}
\item Define the term ``set'' with the label \texttt{set}:
\begin{verbatim}
\newglossaryentry{set}
{
  name={set},
  description={a collection of objects}
}
\end{verbatim}

\item Define the symbol $\emptyset$ with the label
\texttt{emptyset}:
\begin{verbatim}
\newglossaryentry{emptyset}
{
  name={\ensuremath{\emptyset}},
  description={the empty set}
}
\end{verbatim}

\item Define the phrase ``Fish Age'' with the label
\texttt{fishage}:
\begin{verbatim}
\longnewglossaryentry{fishage}
{name={Fish Age}}
{%
  A common name for the Devonian geologic period 
  spanning from the end of the Silurian Period to
  the beginning of the Carboniferous Period.

  This age was known for its remarkable variety of 
  fish species.
}
\end{verbatim}
(The percent character discards the end of line character that would
otherwise cause an unwanted space to appear at the start of the
description.)

\item Take care if the first letter is an extended Latin or
non-Latin character (either specified via a~command such as 
\verb|\'e| or explicitly via the \sty{inputenc} package such 
as \texttt{\'e}). This first letter must be placed in a~group:

\begin{verbatim}
\newglossaryentry{elite}
{
  name={{\'e}lite},
  description={select group or class}
}
\end{verbatim}
or
\begin{alltt}
\verb|\newglossaryentry{elite}|
\verb|{|
  name=\verb|{{|\'e\verb|}lite}|,
  description=\verb|{select group or class}|
\verb|}|
\end{alltt}
\end{enumerate}
(For further details, see the section
\qtdocref{UTF-8}{mfirstuc-manual} in the
\sty{mfirstuc} user manual.)

Acronyms or abbreviations can be defined using
\begin{definition}
\cs{newacronym}\marg{label}\marg{short}\marg{long}
\end{definition}
where \meta{label} is the label (as with the \cs{newglossaryentry}
and the \cs{longnewglossaryentry} commands), \meta{short} is the abbreviation or
acronym and \meta{long} is the long form. For example:
\begin{verbatim}
\newacronym{svm}{svm}{support vector machine}
\end{verbatim}
This defines a glossary entry with the label \texttt{svm}. By
default, the \meta{name} is set to \meta{short} (``svm'' in the
above example) and the \meta{description} is set to \meta{long}
(``support vector machine'' in the above example). If, instead, you
want to be able to specify your own description you can do this
using the optional argument:
\begin{verbatim}
\newacronym
 [description={statistical pattern recognition technique}]
 {svm}{svm}{support vector machine}
\end{verbatim}

Before you define your acronyms, you need to specify which acronym
style to use with
\begin{definition}
\cs{setacronymstyle}\marg{style name}
\end{definition}
where \meta{style name} is the name of the style. There are a number
of predefined styles, such as: \texttt{long-short} (on first use
display the long form with the short form in parentheses);
\texttt{short-long} (on first use display the short form with the
long form in parentheses); \texttt{long-short-desc} (like
\texttt{long-short} but you need to specify the description); or
\texttt{short-long-desc} (like \texttt{short-long} but you need to
specify the description). There are some other styles as well that
use \cs{textsc} to typeset the acronym or that use a footnote on
first use. See the main user guide for further details.

There are other keys you can use when you define an entry. For
example, the \texttt{name} key used above indicates how the term
should appear in the list of entries (glossary). If the term should
appear differently when you reference it in the document, you need
to use the \texttt{text} key as well.

For example:
\begin{verbatim}
\newglossaryentry{latinalph}
{
  name={Latin Alphabet},
  text={Latin alphabet},
  description={alphabet consisting of the letters 
  a, \ldots, z, A, \ldots, Z}
}
\end{verbatim}
This will appear in the text as \qt{Latin alphabet} but will be listed in
the glossary as \qt{Latin Alphabet}.

Another commonly used key is \texttt{plural} for specifying the
plural of the term. This defaults to the value of the \texttt{text}
key with an ``s'' appended, but if this is incorrect, just use the
\texttt{plural} key to override it:
\begin{verbatim}
\newglossaryentry{oesophagus}
{
  name={{\oe}sophagus},
  plural={{\oe}sophagi},
  description={canal from mouth to stomach}
}
\end{verbatim}
(Remember from earlier that the initial ligature \cs{oe} needs to
be grouped.)

The plural forms for acronyms can be specified using the
\texttt{longplural} and \texttt{shortplural} keys. For example:
\begin{verbatim}
\newacronym
 [longplural={diagonal matrices}]
 {dm}{DM}{diagonal matrix}
\end{verbatim}
If omitted, the defaults are again obtained by appending an ``s'' to
the singular versions.

It's also possible to have both a~name and a~corresponding symbol.
Just use the \texttt{name} key for the name and the \texttt{symbol} key
for the symbol. For example:
\begin{verbatim}
\newglossaryentry{emptyset}
{
  name={empty set},
  symbol={\ensuremath{\emptyset}},
  description={the set containing no elements}
}
\end{verbatim}

\section{Using Entries}
\label{sec:useterm}

Once you have defined your entries, as described above, you can
reference them in your document. There are a~number of commands to
do this, but the most common one is:
\begin{definition}
\cs{gls}\marg{label}
\end{definition}
where \meta{label} is the label you assigned to the entry when you
defined it. For example, \verb|\gls{fishage}| will display \qt{Fish
Age} in the text (given the definition from the previous section).

If you are using the \sty{hyperref} package (remember to load it
before \styfmt{glossaries}) \cs{gls} will create a hyperlink to the
corresponding entry in the glossary. If you want to suppress the
hyperlink for a particular instance, use the starred form \cs{gls*}
for example, \verb|\gls*{fishage}|. The other commands described in
this section all have a similar starred form.

If the entry was defined as an acronym (using \cs{newacronym}
described above), then \cs{gls} will display the full form the first
time it's used and just the short form on
subsequent use. For example, if the acronym style is set to
\texttt{long-short}, then \verb|\gls{svm}| will display \qt{support vector
machine (svm)} the first time it's used, but the next occurrence of
\verb|\gls{svm}| will just display \qt{svm}.

If you want the plural form, you can use:
\begin{definition}
\cs{glspl}\marg{label}
\end{definition}
instead of \cs{gls}\marg{label}. For example, \verb|\glspl{set}|
displays \qt{sets}.

If the term appears at the start of a~sentence, you can convert the
first letter to upper case using:
\begin{definition}
\cs{Gls}\marg{label}
\end{definition}
for the singular form or
\begin{definition}
\cs{Glspl}\marg{label}
\end{definition}
for the plural form. For example:
\begin{verbatim}
\Glspl{set} are collections.
\end{verbatim}
produces \qt{Sets are collections}.

If you've specified a symbol using the \texttt{symbol} key, you can
display it using:
\begin{definition}
\cs{glssymbol}\marg{label}
\end{definition}

\section{Displaying a List of Entries}
\label{sec:printglossaries}

In \sectionref{sec:start} I mentioned that there are three options
you can choose from to create your glossary. Here they are again in
a little more detail:

\begin{description}
\item[]\optionlabel1: 

 This is the simplest option but it's slow and if
 you want a sorted list, it doesn't work for non-Latin alphabets.

  \begin{enumerate}
    \item Add \cs{makenoidxglossaries} to your preamble (before you
    start defining your entries, as described in
    \sectionref{sec:defterm}).

    \item Put
\begin{definition}
\cs{printnoidxglossary}[sort=\meta{order},\meta{other options}]
\end{definition}
    where you want your list of entries to appear. The sort
    \meta{order} may be one of: \texttt{word} (word ordering),
    \texttt{letter} (letter ordering), \texttt{case} (case-sensitive
    letter ordering), \texttt{def} (in order of definition) or
    \texttt{use} (in order of use). Alternatively, use
\begin{definition}
\cs{printnoidxglossaries}
\end{definition}
    to display all your glossaries (if you have more than one). This
    command doesn't have any arguments.

    \item Run \LaTeX\ twice on your document. (As you would do to
    make a~table of contents appear.) For example, click twice on
    the ``typeset'' or ``build'' or ``PDF\LaTeX'' button in your editor.
  \end{enumerate}

\item[]\optionlabel2:

   This option uses an application called \texttt{makeindex} to sort 
   the entries. This application comes with all modern \TeX\ distributions, 
   but it's hard-coded for the non-extended Latin alphabet. This process 
   involves making \LaTeX\ write the glossary information to a temporary 
   file which \texttt{makeindex} reads. Then \texttt{makeindex} writes 
   a~new file containing the code to typeset the glossary. \LaTeX\ then 
   reads this file on the next run.

   \begin{enumerate}
    \item Add \cs{makeglossaries} to your preamble (before you start
    defining your entries).

    \item Put
\begin{definition}
\cs{printglossary}\oarg{options}
\end{definition}
    where you want your list of entries (glossary) to appear. (The
    \texttt{sort} key isn't available in \meta{options}.)
    Alternatively, use
\begin{definition}
\cs{printglossaries}
\end{definition}
    which will display all glossaries (if you have more than one).
    This command doesn't have any arguments.

    \item Run \LaTeX\ on your document. This creates files with the
    extensions \texttt{.glo} and \texttt{.ist} (for example, if your 
    \LaTeX\ document is called \texttt{myDoc.tex}, then you'll have 
    two extra files called \texttt{myDoc.glo} and \texttt{myDoc.ist}).
    If you look at your document at this point, you won't see the 
    glossary as it hasn't been created yet.

    \item Run \texttt{makeindex} with the \texttt{.glo} file as the
    input file and the \texttt{.ist} file as the style so that
    it creates an output file with the extension \texttt{.gls}. If
    you have access to a terminal or a command prompt (for example, the
    MSDOS command prompt for Windows users or the bash console for
    Unix-like users) then you need to run the command:
\begin{verbatim}
makeindex -s myDoc.ist -o myDoc.gls myDoc.glo
\end{verbatim}
   (Replace \texttt{myDoc} with the base name of your \LaTeX\
    document file. Avoid spaces in the file name.) If you don't know
    how to use the command prompt, then you can probably access
    \texttt{makeindex} via your text editor, but each editor has a
    different method of doing this, so I~can't give a~general
    description. You will have to check your editor's manual.

    The default sort is word order (``sea lion'' comes before ``seal''). 
    If you want letter ordering you need to add the \texttt{-l}
    switch:
\begin{verbatim}
makeindex -l -s myDoc.ist -o myDoc.gls myDoc.glo
\end{verbatim}

    \item Once you have successfully completed the previous step,
    you can now run \LaTeX\ on your document again.
   \end{enumerate}

\item[]\optionlabel3:

   This option uses an application called
   \texttt{xindy} to sort the entries. This application is more
   flexible than \texttt{makeindex} and is able to sort extended
   Latin or non-Latin alphabets. It comes with \TeX~Live but not
   with MiK\TeX. Since \texttt{xindy} is a Perl script, if you are
   using MiK\TeX\ you will not only need to install \texttt{xindy}, you
   will also need to install Perl. In a~similar way to \opt2, this 
   option involves making \LaTeX\ write the glossary information to 
   a~temporary file which \texttt{xindy} reads. Then \texttt{xindy} 
   writes a~new file containing the code to typeset the glossary. 
   \LaTeX\ then reads this file on the next run.

   \begin{enumerate}
     \item Add the \texttt{xindy} option to the \sty{glossaries}
package option list:
\begin{verbatim}
\usepackage[xindy]{glossaries}
\end{verbatim}

     \item Add \cs{makeglossaries} to your preamble (before you start
     defining your entries).

    \item Put
\begin{definition}
\cs{printglossary}\oarg{options}
\end{definition}
    where you want your list of entries (glossary) to appear. (The
    \texttt{sort} key isn't available in \meta{options}.)
    Alternatively, use
\begin{definition}
\cs{printglossaries}
\end{definition}

    \item Run \LaTeX\ on your document. This creates files with the
    extensions \texttt{.glo} and \texttt{.xdy} (for example, if your 
    \LaTeX\ document is called \texttt{myDoc.tex}, then you'll have 
    two extra files called \texttt{myDoc.glo} and \texttt{myDoc.xdy}).
    If you look at your document at this point, you won't see the 
    glossary as it hasn't been created yet.

    \item Run \texttt{xindy} with the \texttt{.glo} file as the
    input file and the \texttt{.xdy} file as a~module so that
    it creates an output file with the extension \texttt{.gls}. You 
    also need to set the language name and input encoding. If
    you have access to a terminal or a command prompt (for example, the
    MSDOS command prompt for Windows users or the bash console for
    Unix-like users) then you need to run the command (all on one
    line):
\begin{verbatim}
xindy  -L english -C utf8 -I xindy -M myDoc 
-t myDoc.glg -o myDoc.gls myDoc.glo
\end{verbatim}
    (Replace \texttt{myDoc} with the base name of your \LaTeX\
    document file. Avoid spaces in the file name. If necessary, also replace
    \texttt{english} with the name of your language and \texttt{utf8}
    with your input encoding.) If you don't know
    how to use the command prompt, then you can probably access
    \texttt{xindy} via your text editor, but each editor has a
    different method of doing this, so I~can't give a~general
    description. You will have to check your editor's manual.

    The default sort is word order (``sea lion'' comes before ``seal''). 
    If you want letter ordering you need to add the
    \texttt{order=letter} package option:
\begin{verbatim}
\usepackage[xindy,order=letter]{glossaries}
\end{verbatim}

    \item Once you have successfully completed the previous step,
    you can now run \LaTeX\ on your document again.

   \end{enumerate}

\end{description}

For \optsand23, it can be difficult to remember all the
parameters required for \texttt{makeindex} or \texttt{xindy}, so the
\sty{glossaries} package provides a~script called
\texttt{makeglossaries} that reads the \texttt{.aux} file to
determine what settings you have used and will then run
\texttt{makeindex} or \texttt{xindy}. Again, this is a~command line
application and can be run in a~terminal or command prompt. For
example, if your \LaTeX\ document is in the file \texttt{myDoc.tex},
then run:
\begin{verbatim}
makeglossaries myDoc
\end{verbatim}
(Replace \texttt{myDoc} with the base name of your \LaTeX\ document
file. Avoid spaces in the file name.) If you don't know how to use
the command prompt, you can probably access \texttt{makeglossaries}
via your text editor. Check your editor's manual for advice. If you
are using \texttt{arara} then you can just use the directives:
\begin{verbatim}
% arara: pdflatex
% arara: makeglossaries
% arara: pdflatex
\end{verbatim}

The \texttt{makeglossaries} script is written in Perl, so you need
a~Perl interpreter installed. If you are using a~Unix-like operating
system then you most likely have one installed. If you are using
Windows with the \TeX~Live distribution, then you can use the Perl
interpreter that comes with \TeX~Live. If you are using Windows and
MiK\TeX\ then you need to install a~Perl distribution for Windows.
If you are using \opt3, then you need to do this anyway as
\texttt{xindy} is also written in Perl. If you are using \opt2
and can't work out how to install Perl (or for some reason don't
want to install it) then just use \texttt{makeindex} directly, as
described above.

As a last resort you can try the package option \texttt{automake}:
\begin{verbatim}
\usepackage[automake]{glossaries}
\end{verbatim}
This will attempt to use \TeX's \cs{write18} mechanism to run
\texttt{makeindex} or \texttt{xindy}. It probably won't work for
\texttt{xindy} and won't work at all if the shell escape has been
disabled in your \TeX\ distribution. Most \TeX\ distributions will
allow a restricted shell escape, which will only allow trusted
applications to be run. If the \texttt{automake} option is
successful, you will still need to run \LaTeX\ twice for the
glossaries to be displayed.

When sorting the entries, the string comparisons are made according
to each entry's \texttt{sort} key. If this is omitted, the
\texttt{name} key is used. For example, recall the earlier
definition:
\begin{verbatim}
\newglossaryentry{elite}
{
  name={{\'e}lite},
  description={select group or class}
}
\end{verbatim}
No \texttt{sort} key was used, so it's set to the same as the
\texttt{name} key: \verb|{\'e}lite|. How this is interpreted depends
on which option you have used:
\begin{description}
\item[\opt1:] By default, the accent command will be stripped so the
sort value will be \texttt{elite}. This will put the entry in the
\qt{E} letter group. However if you use the
\pkgopt[true]{sanitizesort} package option, the sort value will be
interpreted as the sequence of characters: \verb|\| \texttt{'} \texttt{e}
\texttt{l} \texttt{i} \texttt{t} and \texttt{e}. This will place
this entry before the \qt{A} letter group since it starts with a symbol.

\item[\opt2:] The sort key will be interpreted the sequence of characters:
\verb|{| \verb|\| \verb|'| \texttt{e} \verb|}| \texttt{l} \texttt{i} \texttt{t}
and \texttt{e}. The first character is an opening curly brace
\verb|{| so \texttt{makeindex} will put this entry in the ``symbols'' group. 

\item[\opt3:]
\texttt{xindy} disregards \LaTeX\ commands so it sorts on
\texttt{elite}, which puts this entry in the \qt{E} group.
\end{description}


If the \sty{inputenc} package is used and the entry is defined as:
\begin{alltt}
\verb|\newglossaryentry{elite}|
\verb|{|
  name=\verb|{{|\'e\verb|}lite},|
  description=\verb|{select group or class}|
\verb|}|
\end{alltt}
then:
\begin{description}
\item[\opt1:] By default the sort value will be interpreted as
\texttt{elite} so the entry will be put in the \qt{E} letter group.
If you use the \pkgopt[true]{sanitizesort} package option, the
sort value will be interpreted as \texttt{\'elite} where \'e has
been sanitized (so it's no longer an active character) which will
put this entry before the \qt{A} letter group.

\item[\opt2:] \texttt{makeindex} doesn't recognise \texttt{\'e} as
a~letter so it will be put in the symbols group.

\item[\opt3:] \texttt{xindy} will correctly recognise the sort value
\texttt{\'elite} and will place it in whatever letter group is
appropriate for the given language setting. (In English, this would
just be the \qt{E} letter group.)
\end{description}

Therefore if you have extended Latin or non-Latin characters, your
best option is to use \texttt{xindy} (\opt3) with the \sty{inputenc}
package. If you use \texttt{makeindex} (\opt2) you need to specify the 
\texttt{sort} key like this:
\begin{verbatim}
\newglossaryentry{elite}
{
  name={{\'e}lite},
  sort={elite},
  description={select group or class}
}
\end{verbatim}
If you use \opt1, you may or may not need to use the \texttt{sort}
key, but you will need to be careful about fragile commands in the
\texttt{name} key if you don't set the \texttt{sort} key.

\Tableref{tab:optionsp+c} summarises the main pros and cons of three
options described above.

\begin{table}[htbp]
 \caption{Glossary Options: Pros and Cons}
 \label{tab:optionsp+c}
 {%
 \centering
 \begin{tabular}{>{\raggedright}p{0.3\textwidth}ccc}
   & \bfseries \opt1 & \bfseries \opt2 & \bfseries \opt3\\
   Requires an external application? &
   \no & \yes & \yes\\
   Requires Perl? &
   \no & \no & \yes\\
   Can sort extended Latin
   or non-Latin alphabets? &
   \no\textsuperscript{\textdagger} & \no & \yes\\
   Efficient sort algorithm? &
   \no & \yes & \yes\\
   Can form ranges in the location lists? &
   \no & \yes & \yes\\
   Can have non-standard locations? &
   \yes & \no & \yes
 \end{tabular}
 \par
 }\textsuperscript{\textdagger} Strips standard \LaTeX\ accents so,
for example, \verb|\AA| is treated the same as A.
\end{table}

\section{Customising the Glossary}
\label{sec:glosstyle}

The default glossary style uses the \env{description} environment
to display the entry list. Each entry name is set in the optional
argument of \cs{item} which means that it will typically be
displayed in bold. You can switch to medium weight by redefining
\cs{glsnamefont}:
\begin{verbatim}
\renewcommand*{\glsnamefont}[1]{\textmd{#1}}
\end{verbatim}

By default, a~full stop is appended to the description. To prevent
this from happening use the \texttt{nopostdot} package option:
\begin{verbatim}
\usepackage[nopostdot]{glossaries}
\end{verbatim}

By default, a~location list is displayed for each entry. This refers
to the document locations (for example, the page number) where the
entry has been referenced. If you use \optsor23 described
in \sectionref{sec:printglossaries} location ranges will be compressed.  For
example, if an entry was used on pages~1, 2 and~3, with 
\optsor23 the location list will appear as 1--3, but with \opt1 it
will appear as 1, 2, 3. If you don't want the locations displayed
you can hide them using the \texttt{nonumberlist} package option:
\begin{verbatim}
\usepackage[nonumberlist]{glossaries}
\end{verbatim}

Entries are grouped according to the first letter of
each entry's \texttt{sort} key. By default a~vertical gap is placed
between letter groups. You can suppress this with the
\texttt{nogroupskip} package option:
\begin{verbatim}
\usepackage[nogroupskip]{glossaries}
\end{verbatim}

If the default style doesn't suit your document, you can change the
style using:
\begin{definition}
\cs{setglossarystyle}\marg{style name}
\end{definition}
There are a~number of predefined styles. Glossaries can vary from
a~list of symbols with a~terse description to a~list of words or
phrases with descriptions that span multiple paragraphs, so there's
no ``one style fits all'' solution. You need to choose a~style that
suits your document.

Examples:
\begin{enumerate}
 \item You have entries where the name is a~symbol and the
 description is a~brief phrase or short sentence. Try one of the 
 ``mcol'' styles defined in the \sty{glossary-mcols} package. For example:
\begin{verbatim}
\usepackage[nogroupskip,nopostdot]{glossaries}
\usepackage{glossary-mcols}
\setglossarystyle{mcolindex}
\end{verbatim}

 \item You have entries where the name is a~word or phrase and the
description spans multiple paragraphs. Try one of the ``altlist'' styles. For
example:
\begin{verbatim}
\usepackage[nopostdot]{glossaries}
\setglossarystyle{altlist}
\end{verbatim}
