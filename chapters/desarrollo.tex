%%%%%%%%%%%%%%%%%%%%%%%%%%%%%%%%%%%%%%%%%%%%%%%%%%%%%%%%%%%%%%%%%%%%%%%%%%%
%
% Generic template for TFC/TFM/TFG/Tesis
%
% $Id: desarrollo.tex,v 1.2 2013/12/12 23:06:55 macias Exp $
%
% By:
%  + Javier Mac�as-Guarasa. 
%    Departamento de Electr�nica
%    Universidad de Alcal�
%  + Roberto Barra-Chicote. 
%    Departamento de Ingenier�a Electr�nica
%    Universidad Polit�cnica de Madrid   
% 
% Based on original sources by Roberto Barra, Manuel Oca�a, Jes�s Nuevo,
% Pedro Revenga, Fernando Herr�nz and Noelia Hern�ndez. Thanks a lot to
% all of them, and to the many anonymous contributors found (thanks to
% google) that provided help in setting all this up.
%
% See also the additionalContributors.txt file to check the name of
% additional contributors to this work.
%
% If you think you can add pieces of relevant/useful examples,
% improvements, please contact us at (macias@depeca.uah.es)
%
% Copyleft 2013
%
%%%%%%%%%%%%%%%%%%%%%%%%%%%%%%%%%%%%%%%%%%%%%%%%%%%%%%%%%%%%%%%%%%%%%%%%%%%

\chapter{Desarrollo}
\label{cha:desarrollo}


\begin{FraseCelebre}
  \begin{Frase}
    A fuerza de construir bien, se llega a buen
    arquitecto\footnote{Tomado de ejemplos del proyecto \texis{}.}.
  \end{Frase}
  \begin{Fuente}
    Arist�teles
  \end{Fuente}
\end{FraseCelebre}

\section{Introducci�n}
\label{sec:introduccion-desarrollo}

En este cap�tulo se incluir� la descripci�n del desarrollo del trabajo.

El cap�tulo se estructura en n apartados:...


\section{Desarrollo del sistema de experimentaci�n}
\label{sec:desarr-del-sist}

Blah, blah, blah\ldots


\section{Planteamiento matem�tico}
\label{sec:libr-desarr}

Tambi�n resulta �til poder introducir ecuaciones que se encuentran tanto
en l�nea con el texto (como por ejemplo $\sigma=0.75$), como en un
p�rrafo aparte (como en la ecuaci�n \ref{eq1}). Al igual que ocurre con
las figuras, tambi�n se pueden referenciar las ecuaciones.

\begin{equation}
  \label{eq1}
  p[q_t=\sigma_t|q_{t-1}=\sigma_{t-1}]
\end{equation}

\section{Conclusiones}
\label{sec:conclusiones-desarrollo}

Blah, blah, blah\ldots



%%% Local Variables:
%%% TeX-master: "../book"
%%% End:
