%%%%%%%%%%%%%%%%%%%%%%%%%%%%%%%%%%%%%%%%%%%%%%%%%%%%%%%%%%%%%%%%%%%%%%%%%%%
%
% Generic template for TFC/TFM/TFG/Tesis
%
% $Id: resultados-bare.tex,v 1.1 2015/04/27 17:16:40 macias Exp $
%
% By:
%  + Javier Mac�as-Guarasa. 
%    Departamento de Electr�nica
%    Universidad de Alcal�
%  + Roberto Barra-Chicote. 
%    Departamento de Ingenier�a Electr�nica
%    Universidad Polit�cnica de Madrid   
% 
% Based on original sources by Roberto Barra, Manuel Oca�a, Jes�s Nuevo,
% Pedro Revenga, Fernando Herr�nz and Noelia Hern�ndez. Thanks a lot to
% all of them, and to the many anonymous contributors found (thanks to
% google) that provided help in setting all this up.
%
% See also the additionalContributors.txt file to check the name of
% additional contributors to this work.
%
% If you think you can add pieces of relevant/useful examples,
% improvements, please contact us at (macias@depeca.uah.es)
%
% Copyleft 2013
%
%%%%%%%%%%%%%%%%%%%%%%%%%%%%%%%%%%%%%%%%%%%%%%%%%%%%%%%%%%%%%%%%%%%%%%%%%%%

\chapter{Resultados}
\label{cha:resultados}


\begin{FraseCelebre}
  \begin{Frase}
    % Si quieres ser le�do m�s de una vez, no vaciles en borrar a menudo.
    Rem tene, verba sequentur (Si dominas el tema, las palabras vendr�n
    solas)\footnote{Tomado de ejemplos del proyecto \texis{}.}.
  \end{Frase}
  \begin{Fuente}
    % Horacio
    Cat�n el Viejo
  \end{Fuente}
\end{FraseCelebre}

\section{Introducci�n}
\label{sec:results-introduction}

Blah, blah, blah.

La estructura de este cap�tulo es\ldots


\section{Secci�n 1 del cap�tulo de resultados}
\label{sec:results-1}

Blah, blah, blah.


\subsection{Subsecci�n 1.1 del cap�tulo de resultados}
\label{sec:results-11}

Blah, blah, blah.


\subsection{Subsecci�n 1.2 del cap�tulo de resultados}
\label{sec:results-12}

Blah, blah, blah.




\section{Secci�n 2 del cap�tulo de resultados}
\label{sec:results-2}

Blah, blah, blah.




\section{Conclusiones}
\label{sec:results-conclusions}

Blah, blah, blah.


%%% Local Variables:
%%% TeX-master: "../book"
%%% End:
