%%%%%%%%%%%%%%%%%%%%%%%%%%%%%%%%%%%%%%%%%%%%%%%%%%%%%%%%%%%%%%%%%%%%%%%%%%%
%
% Generic template for TFC/TFM/TFG/Tesis
%
% $Id: conclusiones-bare.tex,v 1.1 2015/04/27 17:16:40 macias Exp $
%
% By:
%  + Javier Mac�as-Guarasa. 
%    Departamento de Electr�nica
%    Universidad de Alcal�
%  + Roberto Barra-Chicote. 
%    Departamento de Ingenier�a Electr�nica
%    Universidad Polit�cnica de Madrid   
% 
% Based on original sources by Roberto Barra, Manuel Oca�a, Jes�s Nuevo,
% Pedro Revenga, Fernando Herr�nz and Noelia Hern�ndez. Thanks a lot to
% all of them, and to the many anonymous contributors found (thanks to
% google) that provided help in setting all this up.
%
% See also the additionalContributors.txt file to check the name of
% additional contributors to this work.
%
% If you think you can add pieces of relevant/useful examples,
% improvements, please contact us at (macias@depeca.uah.es)
%
% You can freely use this template and please contribute with
% comments or suggestions!!!
%
%%%%%%%%%%%%%%%%%%%%%%%%%%%%%%%%%%%%%%%%%%%%%%%%%%%%%%%%%%%%%%%%%%%%%%%%%%%

\chapter{Conclusiones y l�neas futuras}
\label{cha:concl-y-line}


\section{Introducci�n}
\label{sec:introduccion-conclusiones}

Blah, blah, blah.

El cap�tulo se estructura en \ldots apartados \ldots



\section{Conclusiones}
\label{sec:conclusions}

Blah, blah, blah.



\section{L�neas futuras}
\label{sec:lineas-futuras}

Blah, blah, blah.


%%% Local Variables:
%%% TeX-master: "../book"
%%% End:


